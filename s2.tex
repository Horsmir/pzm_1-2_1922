\newpage
\section*{Гибель Европы или торжество империализма?}
\addcontentsline{toc}{section}{2. А.\=,Деборин\---Гибель Европы или торжество империализма?}
\label{sec:2}

\textit{(Доклад, прочитанный 22 января с.\=,г. в Ц.\=,Н.\=,Т.\=,К.).}

Освальд Шпенглер выпустил в свет объемистую книгу под кричащим заголовком <<Гибель Запада>>, выдержавшую в течение непродолжительного времени несколько десятков изданий и произведшую на читающую публику несомненно сильное впечатление. Шпенглер сразу вошел в моду и признаком хорошего тона считается знакомство с его <<философией>>. Появись эта книга до войны, она, быть может, прошла бы незамеченной, несмотря на огромый литературный талант автора. Во всяком случае, никто бы не поверил в возможность <<гибели>> западно\-/европейской культуры, никто из здравомыслящих людей не отнёсся бы серьезно к пророчествам Шпенглера. Но мировая война создала новую обстановку: мировое хозяйство переживает тяжелый кризис, германский империализм лежит в развалинах, династия низвергнута, юнкерство потеряло свое былое значение, среди буржуазии и мещанства царит настроение подавленности и недовольство создавшимся положением. Очевидно, что такая общественная атмосфера является благоприятной почвой для появления пророков и для восторженного приёма их разочарованной <<публикой>>. Вместе с нами, мол, погибает весь мир! Это своего рода идейный реванш германского национализма и империализма над ненавистными врагами\dots

Шпенглер задается целью создать новую философско\-/историческую теорию и с точки зрения этой теории подвергнуть переоценке все культурные ценности и события мировой истории. На основе метафизики истории Шпенглер стремится далее построить целостное всеобъемлющее философское миросозерцание. Каковы же его философско\-/исторический взгляды?

Никто никогда,\---говорит со свойственной ему <<скромностью>> Шпенглер,\---не задумывался над проблемой и структурой истории. Будущим поколениям покажется совершенно невероятным и до крайности наивным наше представление об историческом процессе, как о прямой линии или единой нити, которая тянется от древности до нашего времени и на которую нанизываются, так сказать, исторические события.

Благодаря указанной точке зрения на исторические явления, великие морфологические проблемы истории до сих пор оставались скрытыми от наших глаз. Западная Европа представлялась центром, вокруг которого вращаются великие культуры и события мировой истории. Такое понимание истории свидетельствует только о тщеславии и высокомерии западно\-/европейского человека. Важнейшие события в истории древнего Египта или Китая отступают на задний план, в то время, как события западно\-/европейской истории приобретают для нас первостепенное значение. Мы превращаем нашу культуру в центр мировой жизни. Но с одинаковым же основанием было бы позволительно историку Китая пройти мимо крестовых походов или эпохи ренессанса, т.\=,е. тех событий, которые в нашей истории играют такую выдающуюся роль. Очевидно, что в этой области до сих пор довольствовались <<горизонтом провинцала>>. Конечно, политическому деятелю и социальному критику позволительно оценивать значение других эпох и культур применительно к личному вкусу; мыслитель же должен отрешиться от такого взгляда.

<<Птоломеевская система>> истории должна быть заменена <<Коперниковской системой>>, в которой наряду с античным миром и Западной Европой равноправное место займут Индия, Китай, Египет. Каждая культура образует самостоятельный мир, представляет собою своеобразный организм, который <<неожиданно>> рождается на свет, осуществляет заложенные в нем возможности и умирает. Культура есть внешнее выражение определенного строя свойственной ей души. Кто хочет познать истинную природу и своеобразный характер культуры, тот должен проникнуть в её душу, составляющую сущность культуры. Каждый культурный организм представляет собою самостоятельную замкнутую систему, живет своей особой жизнью и создаёт свои ценности, свою науку, искусство, технику, социальные н политические учреждения. Никакой преемственности, никакой внутренней связи между культурами не существует. Античный мир погиб безвозвратно, как безвотвратно погибнет Западная Европа, никому не передав своего наследства. Человечества, как единого субъекта мировой истории не существует. Каждый культурный организм имеет свою историю, ничего общего не имеющую с исторической жизнью других культур. Такая точка зрения, говорит Шпенглер,\---вскрывает перед нами <<всё богатство красок, движений, света, которое до сих пор не открывалось ни одному духовному взору>>. Всё, что до сих пор говорилось и писалось о проблеме времени, пространства, движения, собственности является ошибочным, потому что предполагалось, что всем людям присущи одинаковые формы сознания. А между тем <<форм сознания>> существует столько же, сколько отдельна культур или лежащих в их основе душ.

Культурно\-/психический тип рождается из хаоса с определенным религиозным настроением, которым проникнута вся его творческая Деятельность. Всякая наука вплоть до самых точных, как математика или механика, предполагает как свою основу и источник религию.

Западно\-/европейская механика,\---вещает наш оракул, является отражением христианских догматов. Современное естествознание есть не что иное, как функция именно нашей культуры. Оно не только предполагает религию, как источник, из которого оно возникло, но оно постоянно от нее зависит и ею обусловливается. \emph{Без религии нет культуры.}

Жизнь культурного организма состоит в непрестанной борьбе духа или души против внешнего материального мира. Душа стремится к осуществлению своей <<идеи>>, своих внутренних возможностей\---к объективации переживаний в царстве протяженного. Душа\---есть чистое становление; сущность ее состоит в непрестанном творчестве, непрекращагощейся деятельности. Продукты же этой деятельности составляют совокупность символов, данных в опыте, но представляющих только видимость, внешнее отражение лежащей позади них сущности. Какое\-/нибудь украшение, напр., на саркофаге является\---символическим выражением определенного душевного настроения, которое доступно только людям данной культуры. Начиная от телесных выражений\---физиономии, осанки, жестов, как отдельного лица, так и целых классов и народов\---вплоть до форм хозяйственной политической и общественной жизни или мнимо\-/вечных и общеобязательных форм познания\---всё это представляет собою не что иное, как своеобразные символы или формы выражения данной души. Совокупность символов составляет физиономию культуры, чувственный образ души. Но этот чувственный образ доступен <<восприятию>>и познанию людей принадлежащих к данной культуре. Мы способны постичь душу египтян или индусов самым несовершенным образом. Когда мы, люди западной культуры, принимаемся интерпретировать египетскую или греческую статую, мы прибегаем к помощи нашего внутреннего опыта, который совершенно не соответствуюет опыту и переживаниям египтянина или грека.

Первичным символом всякой культуры является пространство, которым определяется характер и природа всех остальных символов, т.\=,е. всех внешних форм культуры. Каждый культурно\-/психический тип переживает и чувствует пространство по\-/своему. Понятие пространства тесно связано с идеей смерти, ибо страх смерти равнозначущ со страхом перед пределом\---перед пространством. <<Геометрическое>> пространство грека или аполлоновской души противоположно <<аналитическому>> пространству фаустовской души, т.\=,е. западноевропейского человека. <<Геометрическая>> душа античного человека стремится к покою и сосредоточивает свое внимание на конечном, мгновенном и телесном. Фаустовская же душа характеризуется стремлением к бесконечному и волей к власти. Эти основные свойства проникают все феномены культуры. Отсюда коренное отличие античной математики, физики, механики, искусства от современных западно\-/европейских.

Античный дух создает механическую \emph{статику;} западно\-/европейский\---механическую \emph{динамику.} В математике фаустовский дух выражается в понятии функции, которая растворяется в ряде процессов. Аполлоновская душа имеет представление о числе, как об определенной и конечной величине; это в известном смысле пространственное понятие. Античный мир вкладывал в понятие движения представление 	<<перемещений>>, между тем, как фаустовский дух в идее движения видит, <<направленность>> к бесконечному. Три механические системы\--- статика, химия и динамика соответствуют различию трех культурно\-/психических типов: аполлоновскому, магическому (арабскому) и фаустовскому. Им же соответствуют и три различные математические концепции: евклидова геометрия, алгебра и анализ. В области искусства мы имеем: статую, арабеску и фугу. В физике: механический порядок состояний, скрытых сил и процессов. Словом, аполлоновская душа отличается статическим характером, фаустовская\---динамическим.

Атомы древних материалистов Левкиппа и Демокрита в соответствии с основным принципом аполлоновской души различаются друг от друга только формой и величиной; это чисто пластические элементы. Атомы современной западно\-/европейской физики\--- центры сил; их неделимость имеет особый имматериальный смысл. Основным для западно\-/европейского человека является понятие силы, движения, прогресса, воли к власти; основным для античного\--- понятие формы, покоя, ограниченности.

Единством души определяется внутренняя связь всех феноменов культуры. <<Между дифференциальным исчислением и династическим государственным принципом Людовика XIV, между античным городом (<<полисом>>), как формой государства и эвклидовой геометрией, между пространственной перспективой, западно\-/европейской живописью и преодолением пространства при помощи железных дорог, телефонов и дальнобойных орудий, между контра пунктической инструментальной музыкой и кредитной системой существует глубокая связь форм>>. Единство метафизики, математики, религии, физики, искусства, техники определённой культуры коренится в единстве её <<души>> или <<идеи>>. Насколько Шпенглер в этой части своего мировоззрения близок к Гегелю осведомлённому читателю ясно без особых разъяснений.

Наряду с принципом единства <<форм>>, образующих в своей совокупности образ или \emph{физиономию} культуры, составляющую специфический предмет исследования историка или философа культуры, особое место в исторической теории Шпенглера занимает принцип аналогии или точнее гомологии. При помощи этого принципа, мы получаем возможность сопоставлять механические, физические, политические и т.\=,д. системы различных культур и определять их сравнительное значение для данных культурных организмов. При помощи принципа аналогии мы в состоянии установить безошибочно \emph{возраст} культуры, еще не закончившей цикла своего развития, сравнивая ее с культурой, отошедшей уже в вечность. Историческая имеет своей задачей сравнительное изучение соответствующих возрастов и связанных с ними форм. Если принять во внимание, что продолжительность жизни каждой культуры определяется Шпенглером в тысячу лет, и что каждый возраст также имеет вполне определенную, для всех культур одинаковую, продолжительность, то ясно, что принцип <<хронологии>> в соединении с идеей <<возрастов>> даёт возможность легко предвидеть исторические события. Сопоставляя античный стоицизм с современным социализмом, Шпенглер на основании <<хронологии>> приходит к выводу относительно предстоящей гибели западно\-/европейской культуры, так как социализм, видите\-/ли, представляет собою такой же симптом дряхлости нашей культуры, как стоицизм\---античной.

На рубеже девятнадцатого столетия Европа вступила в старческий возраст и с тех пор быстрыми шагами приближается к смерти. Предстоящая Европе смерть является неизбежным следствием переживаемой ею эпохи цивилизации. Всякая культура переживает три фазиса: этнографическое состояние (<<первичное состояние хаоса>>, как выражается Шпенглер), собственно культурное и цивилизационное. Первичный период, составляющий детство культурного организма, чрезвычайно продолжителен; это период подготовки, формирования психического типа и собирания сил. Собственно\---культурный период\---это эпоха напряженного и интенсивного творчества; в этот именно период создаются глубокие метафизические системы, рождаются великие произведения искусства, закладываются прочные основы общественной и политической организации,\---словом, зто период зрелости культуры и расцвета всех жизненных сил её. За культурным периодом следует эпоха цивилизации, которая продолжается обычно двести\==триста лет. Это период дряхлости организма, приближающегося к смерти.

Возраст имеет \emph{органическое} значение для культурного типа; он налагает вполне определенную печать на всю творческую деятельность народа. Возраст обнаруживается во всех его творениях. И наоборот: по характеру деятельности культурного типа мы имеем возможность судить о переживаемом им возрасте: Так, <<возраст>> и <<хронология>> приобретают для Шпенглера мистическую ценность и значение.

Дуализм души и тела (совокупности символов) культуры ведет к признавию двух различных способов познания\---рассудочного и интуитивного. Рассудком мы способны постигать только <<видимость>> вещей. Для проникновения же в лежащую позади явлений сущность вещей необходим особый орган\---<<божественное созерцание>>, которое является даром пророка, но которое не дано обыкновенным смертным.

Всякая культура проникнута роковым дуализмом, неразрешимыми противоречиями. Душа есть жизнь, а жизнь\---вечный поток, становление; она органически связана с \emph{временем} и с понятием \emph{судьбы.} Судьба тожественна с переживаемым каждым человеком чувством времени, которое означает неизбежность всего совершающегося, предопределенность и необратимость жизненного процесса. Тайна судьбы скрывается в складках временя. Таким образом судьба и время\---формы интуитивного опыта.

Тело культуры\---совокупность символов\---познается нами посредством категорий пространства и причинности, ибо <<символы>>\---застывшее мертвое бытие. Но так как <<символы>> являются продуктами творческой деятельности души, то время порождает пространство, судьба\---причинную необходимость, дух\---мир, история\---природу (которая ведь есть не что иное, как функция определенной культуры, т.\=,е. её души), жизнь\---смерть. В силу этой противоположности существуют две формы космической необходимости: органическая и механическая. Механическая или неорганическая логика, т.\=,е. наука рождается в результате борьбы с органической логикой. Механизирующий рассудок восстаёт против непостижимой и неумолимой судьбы в целях укрощения и подчинения её себе.

Органический период культуры характеризуется признанием необходимости судьбы и изъявлением ей покорности. Человек во всем видит волю Бога. Стремление же человека при помощи созданной им науки приобрести власть над судьбой, над непостижимым означает бунт рассудка против воли Провидения. В этом повороте от религии и метафизики к положительной науке и заключается, по мнению Шпенглера, переход культуры в цивилизацию. Цивилизация, т.\=,е. научное рассудочное познание, <<убивает>> идею судьбы и непознаваемого,\---жалуется Шпенглер. Религия и метафизика исходят из признания непостижимого я требуют безусловного ему подчинения. Народ должен подчиниться слепому року и олицетворяющим его силам\---в том числе и общественным. Научное познание отвергает непознаваемое и непостижимое и тем самым подчиняет <<судьбу>> человеческой воле. <<Дерзость>> разума заключается в том, что он вскрывает внутреннюю связь явлений, берет свою судьбу в собственные руки и <<свергает>> стоящие над жизнью таинственные силы.

Культура, как и цивилизация, связаны с определенным общественным укладом. Цивилизация продукт мирового города, культура же связана с землёй, с деревней. В эпоху цивилизации на историческую арену выступает <<паразит>>, безформенная масса, лишенная всех традиций. Творцом же культуры является  сросшийся с землей <<народ>>. Некогда борьба за идеальное постижение смысла жизни разыгрывалась на почве религии и метафизики между <<почвенным>> духом крестьянства, т.\=,е. дворянства и духовенства (как вам, читатель, нравится это отожествление крестьянства с дворянством и духовенством?) и светским духом старых городов. Современный обитатель города рационалист, атеист и радикал; он с отвращением относится к религии и метафизике, враждебно настроен по отношению к <<крестьянству>>. Он отказывается прията мир и жизнь, как ниспосланную Богом судьбу. Решающую роль в современной жизни играют <<народные массы>>, которые ведут ожесточенную борьбу с \emph{культурой, т\=,е. с дворянством, церковью, династиями и привилегиями.} Таким образом, <<культура>>, освобожденная от метафизического тумана, которым Шпенглер её окутывает, предстает перед нами в образе <<алтаря и престола>>.

Эпоха цивилизации характеризуется рационализмом, натурализмом и материализмом. Рационалистическая критика всех основ жизни и всеуравнивающий натурализм требуют уничтожения исторических различий между привилегированными и закрепощёнными (какие ужасы!), замены существующей государственной организации справедливым общественным строем. В связи со всем этим возникает и новая плебейская мораль социализма, которая ставит перед собою практические задачи и имеет целью преобразование жизненных форм. <<Трагическая>> мораль полнокровной культуры сознает <<тяжесть>> бытия и неизбежность судьбы, в то время как социализм строит стратегические планы в целях <<обхода>> судьбы. Социалистическая или плебейская, как презрительно называет ее Шпенглер, этика проникнута гуманностью, она проповедует всеобщее братство, счастье большинства и мир между народами. Если такие ужасные вещи имеют место в действительности, а этого отрицать нельзя,\---то ясно, что мы накануне светопреставления, что культура погибает, что Европа сгнила, что она переживает агонию и бьется уже в предсмертных судорогах\dots

\begin{center}
 \noindent\textasteriskcentered\ \textasteriskcentered\ \textasteriskcentered
\end{center}

Итак, носителями и творцами культуры являются дворянство, духовенство и король. Даже буржуазия в качестве культурного фактора для Шпенглера не существует. Поэтому нет ничего удивительного, что органический период культуры оканчивается с великой французской революцией. Правда, формула Шпенглера совершенно не соответствует исторической действительности, ибо рационализм, механизм, натурализм и отчасти материализм являются господствующими умственными течениями XVI, XVII и XVIII столетий. Но какое дело Шпенглеру до фактов. Ведь <<факты>>, эмпирическая действительность являются лишь <<символом>> метафизической сущности, лежащей позади явлений. Для уразумения истинного смысла действительности необходимо <<божественное созерцание>>, которое способно постигнуть <<более глубокое значение>> данных в эмпирическом мире явлений. Так что <<факты>> с метафизической точки зрения иногда означают нечто совершенно другое, чем нам, обыкновенным смертным, это кажется. При таких условиях рационализм и натурализм в метафизическом аспекте, т.\=,е. по истинной своей сущности, превращаются в свою противоположность. С этой точки зрения, какой\-/нибудь Гегель в метафизическом свете может превратиться в самого яркого выразителя натурализма и механизма. Так оно в действительности и происходит у Шпенглера. Но ясно, что на этой почве спор с Шпенглером совершенно невозможен. Его <<метафизику>> истории остается или целиком отвергнуть, или же принять на веру. Блаженны верующие! Но мы к их числу не принадлежим.

Шпенглер поставил себе определенную задачу, которая сводится к утверждению принципа народности в науке, искусстве, политике, технике. Этот принцип народности должен оправдать специальные притязания прусского империализм. В этих целях воздвигается нашим мыслителем китайская стена между отдельными культурами или культурно\-/психическими типами. Нет общезначимых, объективных истин, на всем лежит печать <<народности>> нет человечества, по отношению которого у отдельных индивидов или народов существовали бы какие\-/либо обязанности. Обязанности существуют лишь по отношению к своему типу или народу. Словом, все национально, в том числе и социализм.

Проблема цивилизации совпадает с проблемой пролетариата и социализма. Пролетариат\---это обозлённая, безформенная масса, полная ненависти к культуре, религии и метафизике. Пролетариат создает себе особое социалистическое мировоззрение, основанное на рационализме, сенсуализме и материализме. Для него преобледающее, даже исключительное, значение приобретают практические задачи преобразования общества. Выступление рабочего класса на историческую арену знаменует собою начало \emph{гибели культуры.} Собственно говоря, в этой <<философии>> нет ничего нового; реакционеры всех стран видят в пролетариате того варвара, который пришел разрушить культуру и уничтожить все ее ценности.

Марксизм является общепризнанной идеологией международного рабочего движения. И поэтому естественно, что Шаенглер на всём протяжении своей книги ведет открытую борьбу с материалистическим вниманием истории. Маркс,\---говорит Шпенглер,\---силен в критике существующего, но он поверхностен и крайне безпомощен, как творец. Это особенно ясно видно на отношении Маркса к истории, которая рассматривается им как эволюционный процесс. Человечество таких целей не имеет. Счастье людей\---нелепая идея. Мировое гражданство\---жалкая фраза. <<Мы люди определенного века, определённой нации, круга, типа>>, И эта наша принадлежность к определенному типу является необходимым условием, при котором наша жизнь приобретает <<смысл и глубину>>. Чем больше Мы сознаём эту национальную ограниченность, тем большее значение и ценность приобретает наше деятельность.

Социализм так же национален, как искусство, математика и философия; рабочих движений существует столько же, сколько отдельных живых рас. Они относятся друг к другу с такой же ненавистью, как и соответствующие народы. В минувшую войну против Германии наряду с буржуазией Антанты выступил и псевдосоциализм стран Согласия, но это была война против истинного, т.\=,е. прусского социализма. Прусско\-/германский социализм имеет своим врагом не немецкий капитализм, который давно уже проникся социалистическим духом, а антантовский лжесоциализм. Вожди официального немецкого социализма,\---жалуется Шпенглер,\---не понимают той простой истины, кроме прусского социализма никакого другого нет. Французский социализм основан на элементарном чувстве социальной мести; это социализм саботажа и путшей; английский социализм не что иное, как особая форма капитализма. Английские и французские рабочие являются самыми яркими представителями своей расы. В отличие от французского и английского только один прусский социализм является мировоззрением. Пруссак\---прирожденный социалист\dots

Прусский инстинкт гласит: власть принадлежит государству, личность находится на службе у государства, а король первый его слуга. Эта концепция составляет сущность, так наз., авторитарного социализма, который по самому существу своему антиреволюцнонен и антидемократичен. Авторитарный социализм родился в Пруссии еще в XVIII столетии, и задача нашего века состоит в том, чтобы вернуться к идеалу ХVIII века. Но зловредный марксизм спутывает все карты. Марксизм превратил одну часть немецкого народа, состоящего из крестьян и чиновников, в <<четвертое сословие>>, а другую преобладающую часть в третье сословие, избрав последнее объектом классовой борьбы.

Марксистские партии сыграли роковую роль в революции 1917 г., ибо коренная ошибка их заключалась в том, что они стремились к осуществлению того, что в Германии давно уже является действительностью. В самом деле, что такое социализм,\---спрашивает Шпенглер? И отвечает: социализм\---это политический, хозяйственный и социальный инстинкт реалистически настроенных народов. В этом инстинкте живет старая фаустовская воля к власти, воля к безусловному мировому господству в военном, экономическом и умственном отношениях. Этот инстинкт получил самое яркое выражение в факте мировой войны и идее социальной революции. Стоящая перед нашей цивилизацией задача сводится к необходимости сковать посредством фаустовской техники всё человечество в единое целое. В этом и заключается внутренний смысл современного империализма. Таким образом, Шпенглер приходит к отожествлению социализма с империализмом. Пруссаки являются воплощением фаустовского духа в его самой чистой и совершенной форме. Стало быть, пруссаки призваны выполнить миссию социализма и империализма.

Прусская идея, в отличие от английской, заключается в принципе сверхличного <<товарищества>>. Англия\---классическая страна капитализма, Пруссия\---классическая страна социализма. Гогенцоллерны являются носителями <<общинного>> духа и слугами государства. Островное положение Англии способствовало развитию личного начала и сделало возможным существование народа без прочной государственной организации. Личное начало враждебно порядку и проявляется в жестоrой эксплуатации бjлее слабых народов и классов.

В пруссаке же сильна и жива идея государства и труда. Каждый класс проникнут глубоким сознанием своего социального призвания и чувством преданности целому. Принадлежность к тому или иному сословию обусловливается не богатством, как в Англии, а <<рангом>>, ибо для пруссака труд имеет самостоятельную нравственную ценность; труд для него призвание. Прусская или социалистическая этика учит, что в жизни следует стремиться к исполнению долга, а не в обогащению, как этого требует английская капиталистическая этика.

Социальные различия в обоих странах имеют неодинаковый характер. Низшие классы в Англии не имеют никакого значения и не играют никакой роли. Положение же трудящихся классов в Пруссии определяется возможностью достигнуть любого <<ранга>>. Социалистический строй имеет своим основанием \emph{авторитет.} В этом смысле Пруссия\---идеал социалистического порядка. Пруссак презирает богатство и подчиняется авторитету своего вождя. Целью хозяйственной деятельности пруссака является благо и процветание целого. Эта экономическая идея теснейшим образом связана с принципом государственного авторитета. Каждый член общества получает свое хозяйственное, так сказать, назначение от организованного авторитета государства. Гогенцоллерны и выполняли роль колонизаторов и организаторов общества и в этом же духе воспитали немцев.

Английская же хозяйственная система построена на началах эксплуатации и грабежа. Марксизм является порождением английского капитализма и вполне к нему применим, но только к нему, ибо только в Англии происходит ожесточённая борьба между капиталом и трудом. Вместе с тем не следует забывать, что английское рабочее движение ничего общего с социализмом не имеет. Английский социализм проникнут целиком духом капитализма; в то же время прусский капитализм по существу своему социалистичен. Прусский капитализм давно уже принял социалистические формы в смысле своеобразного государственного порядка и государственного управления хозяйством. Таким образом, Англия и Пруссия противостоят друг другу, как две непримиримых хозяйственные системы. Так как оба народа являются выразителями фаустовского принципа, то оба они будут стремиться навязать свою идею и свою волю всему остальному миру; борьба между обоими народами будет продолжаться до тех пор, пока одна из сторон не одержит окончательной победы над другой. Проблема, которая должна быть решена в общемировом масштабе, может быть формулирована в следующих словах: должно ли мировое хозяйство стать мировой организацией, т.\=,е. подчиниться прусскому принципу или же оно примет всеобщую форму мировой эксплуатации, т.\=,е подчинится английскому принципу.

Вместе с английским принципом уничтожению подлежит и марксовский псевдо\-/интернационал, на место которого должен стать истинно\-/прусский интернационал, который, однако, возможен только в результате победы идеи одной расы над всеми другими. Надо помнить, говорит Шпенглер, что в действительной жизни нет места примирению или компромиссам. Здесь возможна только победа или смерть\---смерть народов и культур. К смерти приговорены французы и англичане, ибо культура их давно сгнила, а творческие силы этих народов давно иссякли. Это дряхлые организмы, которым пора на покой. Другое дело немцы. Немцы\---народ молодой и сильный; <<наша гибель лежит в \emph{туманной дали отдаленного будущего}>>. Живя в текущем столетии, будучи вплетены и связаны всем существованием своим с фаустовской цивилизацией, мы, немцы, имеем перед собой величайшие задачи, массу еще неосуществленных возможностей. Отсюда ясно, что гегемония над миром должна принадлежать только немцам, самому молодому и самому сильному из фаустовских народов. Немцы единственный народ, которой призван осуществить идею мировой цивилизации, мирового государства, идею \emph{интернационала.}

<<\emph{Истинный интернационал,} говорит Шпенглер>>, это \emph{империализм.} Стоящая перед интернационалом задача заключается в завоевании фаустовской цивилизации при помощи одного организующего принципа. Европа стоит перед опасностью стать добычей народа\-/эксплуататора, т.\=,е. англичан, которые обратят всех в рабство. При таких условиях Пруссия призвана спасти мир. Но Пруссия в состоянии будет выполнить свою миссию только при условии освобождения немецких рабочих от иллюзий марксизма и марксовского интернационализма. Иначе говоря, Пруссия в состоянии будет осуществить свои империалистические стремления только притом условии, если немецкие рабочие проникнутся идеологией империализма и заключат союз с <<старо\-/прусскими силами>>, т.\=,е. с монархией и юнкерством\---с этими истинными носителями и творцами культуры.

Но так как немецкие рабочие уже достаточно <<развращены>> цивилизацией, то необходимо им дать известную <<компенсацию>> в виде истинно\-/прусского социализма.

Перед проницательным взором Маркса, говорит Шпенглер, предстала во всей своей ясности перспектива гибели Европы в случае победы английского принципа. Поэтому он со всей силой обрушивается на английское понятие частной собственности. Но формулировка Маркса носит чисто отрицательный характер: экспроприация экспроприаторов. Несмотря, однако, на эту отрицательную формулировку, в ней заключается все же прусский принцип: <<уважение>> перед собственностью и стремление содержащуюся в ней мощь передать общественному целому\---государству. Эта проблема носит, название \emph{социализации.} Но оказывается, что и тут Маркса предупредили Гогенцоллерны, ибо прусское правительство, начиная от Фридриха Вильгельма I вплоть до последнего времени проводило политику социализации. И кто <<испортил>> это великое дело Гогенцоллернов? Опять таки ненавистные Шпенглеру марксисты.

Марксовское понимание социализации,\---говорит Шпенглер, в корне ошибочно, ибо социализация вовсе не означает обобществления или огосударствления собственности путем отчуждения её. Социализация\---вопрос не номинального владения, а чисто \emph{техническая проблема управления.} Старо\-/прусская идея социализации и заключалась в том, чтобы при тщательном соблюдении нрава собственности и наследования всю совокупность производительных сил \emph{формально подчинить законодательству.} Таким образом социализация сводится к постепенному превращению рабочего в хозяйственного чиновника, предпринимателя\---<<в ответственного управляющего с широкими полномочиями>>, <<собственности\---в род наследственного лена в смысле старых времён, связанного с известной суммой прав и обязанностей>>. Социалистические способности прусского чиновника являются гарантией возможности осуществления <<социализации>>. Тип прусского чиновника\---единственного в мире\--- воспитан Гогенцоллернами. Немецкий рабочий должен стать таким <<чиновником>>, ибо <<государство будущего это государство чиновников>>.

Идеал консервативной партии,\---говорит Шпенглгр,\---совпадает при таких условиях с идеалом пролетариата, ибо оба стремятся к полному огосударствлению хозяйственной жизни законодательным путем. Но такая задача может быть выполнена только \emph{монархом,} преданным <<традициям своего рода и мировоззрению своего призвания>>. Он будет играть роль третейского судьи в спорных случаях, ибо монарх стоит над партиями и классами и несёт заботу о государственном целом. В социалистическом государстве он будет производить отбор <<руководителей>> по моральным их качествам. Словом, монарх\---единственная опора и защита против торгашества (читай: англичан). Республика означает продажность государственной масти. <<Президент, премьер\-/министр или народный уполномоченный являются креатурами партии, а партия\---креатура тех, кто оплачивает>>.

Судьбу Европы, а с ней и всего мира, решит Германия. Надо имнить, что пруссачество и социализм\---вещи тожественные. Марксизм и классовый эгоизм повинны в том, что социалистический пролетариат и консервативная партия до сих пор враждовали между собою. <<Объединение их означает осуществление идеи Гогенцоллернов и одновременно освобождение рабочего класса>>. Пролетариат должен покончить с иллюзиями марксизма и псевдо\-/интернационализма. Для рабочего класса существует только прусский социализм, или его вовсе нет,\---пугает Шпенглер рабочих. Рядом с этим идет увещевание консерваторов или юнкеров. Консерваторы также должны освободиться от классового эгоизма. Они должны понять, что демократия\---политическая форма нашей эпохи, как бы мы ни относились к ней по существу. Если консерваторы не хотят \emph{погибнуть,} они должны сознательно стать на сторону \emph{социализма.} Демократия не должна отпугивать консерваторов, ибо речь идет не об английско\-/французких формах её, а о прусской. Наша свобода состоит в освобождении хозяйственного произвола отдельного лица.

Заканчивается эта империалистическая философия пламенным призывом к молодому поколению, которому Шпенглер советует стать мужами, воспитать себя для предстоящей великой задачи. Нам,\---говорит Шпенглер,\---необходим класс <<социалистических господских натур>>. \emph{Социализм означает власть над миром, осуществление империилистического интернационала.} Путь к нему лежит через объединение немецких рабочих с <<лучшими носителями>> старо\-/прусской государственности.

\begin{center}
 \noindent\textasteriskcentered\ \textasteriskcentered\ \textasteriskcentered
\end{center}

Итак, основной сногсшибательный тезис о неизбежной гибели западно-европейской культуры предстал перед нами теперь в истинном свете. Оказывается, Европа может быть еще спасена Пруссией гибель которой <<лежит в туманной дали отдаленного будущего>>. \emph{Победа прусского империализма может предотвратить гибель западно-европейской культуры.}

\emph{Теоретическое} же обоснование тезиса о гибели культуры до чрезвычайности легковесно. В защиту своего положения Шпенглер приводит два довода: один\---метафизический, а другой\---эмпирический. Метафизический довод вытекает из всей концепции Шпенглера о культуре, как организме, обладающем самобытной душой. Душа обладает \emph{ограниченными возможностями;} силы ее истощаются после осуществления этих <<возможностей>>. Душа, воплотив свою <<идею>> в действительность, дряхлеет и умирает. Но это соображение не может быть принято всерьез, по крайней мере, теми, которые не обладают способностими проникнуть в природу <<души>> культуры. О другой стороны, оно опровергается эмпирическими фактами, которые свидетельствуют о сравнительной долговечности отдельных культур, в роде китайской.

Что же касается второго <<эмпирического>> довода, то он основан на аналогии между античным миром и западной Европой. Западная культура в лице социализма переживает, мол, античный стоицизм, т.\=,е. период дряхлости. Аналогический метод сам по себе чрезвычайно ненадежен, но его теоретическая ценность равна почти нулю, когда на основании \emph{одной аналогии} между двумя случайно выбранными явлениями делается заключение, претендующее на значение исторического закона. Это формальная сторона вопроса. По существу же, между стоицизмом и социализмом существует столько же общего, сколько между Шпенглером и\dots Марксом. Стоицизм был проникнут глубокой религиозностью, пассивностью и фатализмом. Стоицизм, подобно Шпенглеру, требовал от человека подчинения \emph{судьбе.} Между тем, как современный научный социализм рационалистичен, атеистичен и активен, как это признал тот же Шпенглер, и стремится превратить человека из раба в господина <<судьбы>>. Не в этом ли залог победы социализма и \emph{спасения культуры?} Да, старая культура, если под ней понимать религию и метафизику, <<погибает>>, как <<погибают>> те общественные классы, которые являлись ее носителями, поборниками и творцами. Но следует ли отсюда, что религия и метафизика составляют исключительное содержание всякой культуры? Конечно, нет! Шпенглер утверждает, что всякая культура имеет своим \emph{источником} религию и что, стало быть, там, где нет религии и метафизики, нет науки, нет искусства, нет культуры вообще. Но это одно из тех многочисленных изречений Шпенглера, которые не имеют под собою никакой почвы. Ибо современная культура родилась и достигла расцвета именно в борьбе с религией и метафизикой. Содержание культуры меняется от эпохи к эпохе, сама же культура остается и делает все новые и новые завоевания. Социализм стремится не к разрушению культуры, а к <<завоеванию>> её и к дальнейшему её развитию, вложив в неё новое содержание. Стало быть, речь может идти о <<гибели>> определенного содержания культуры, но не культуры вообще.

Чем определяется характер культуры? Природой, лежащей в ея основании души, отвечает Шпенглер. Но так как каждый народ и даже сословие (не говоря уже об отдельном человеке) в свою очередь обладают <<душой>>, то единая душа культуры распадается на совокупность отличных друг от друга душ. Таким образом, мы получаем единство во множестве и множество в единстве. В <<пределах>> Фаустовской души, как мы видели, существует прусская и английская душа, существуют различные <<инстинкты>>, которые противоположны друг другу и в то же время составляют все же единство. Очевидно, что заключение отсюда к единой \emph{<<общечеловеческой душе>>} (или <<абсолютной душе>>) напрашивается само собой. Однако, Шпенглер стоит твердо на почве культурно\-/психических типов, абсолютно разнородных и чуждых друг другу. Но эта логическая непоследовательность нашего философа искупается и оправдывается, повидимому, последовательностью <<психологической>>, последовательностью националиста и империалиста. В конечном счете для Шпенглера истинной <<реальностью>> является <<душа>> его народа.

И в этом отношении, как, впрочем, и во всей своей философско\-/исторической концепции, он остается верным учеником своего учителя\---И.\=,Я.\=,Данилевского\footnote{Данилевский в свою очередь заимствовал свою теорию у немецкого историка Гейнриха Риккерта (см.\=,его соч. <<Lehrbuch der Welgeschichte in organischer Darstellung>>, Leipzig, 1857).}). Этот идеолог русского национализма еще в 1869\=,г. изложил во всех подробностях и теорию культурно\-/исторических типов, и идею морфологии, и принцип аналогии, и теорию возрастов\---словом предупредил Шпенглера во всех мелочах. Не имея возможности остановиться подробнее на проведении паралели между обоими писателями, мы все же позволим себе вкратце изложить основные мысли Данилевского в доказательство того, что Шпенглеровская <<Америка>> давно уже открыта. Данилевский, как и Шпенглер, исходит из мысли о существовании обособленных культурно\-/исторических типов, составляющих те <<реальности>>, с которыми имеет дело история. Человечество\---отвлеченное понятие, а не реальность, и говорить о единстве человечества, как субъекта истории, об общей теории развития политических обществ, о преемственности начал культуры, науки, искусства в поступательном ходе развития народов абсолютно не приходится. Гегелевская схема развития человечества, согласно которой история отдельных народов является как бы отрезком единой линии, последовательными ступенями человеческого прогресса, является в корне ошибочной. Общей культуры нет, существует только развитие отдельных культурно\-/исторических типов, которые представляют собою ряд параллельных линий, никогда не встречающихся и нигде не сталкивающихся. И Данилевский почти теми же словами, что и Шпенглер, обрушивается на историков за то, что у них <<самая общая группировка всех исторических явлений и фактов состоит в распределении их на периоды древней, средней и новой истории>>\footnote{ И.\=,Я.\=,Данилевский. Россия и Европа, 3\-/е издан. 1888\=,г., стр.\=,82.}), и что <<судьбы Европы или германо\-/романского племени были отождествлены с судьбами всего человечества>>\footnote{Там же, стр.\=,85.}). <<Деление истории на древнюю, среднюю и новую, хотя бы и с прибавлением древнейшей и новейшей,\--говорит Данилевский,\---или вообще деление по ступеням развития\---не исчерпывает всего богатого содержания её. Формы исторической жизни человечества, как формы растительного и животного мира, как формы человеческого искусства (стили архитектуры, школы живописи), как формы языков (односложные, приставочные, сгибающиеся), как проявление самого духа, стремящегося осуществить типы добра и красоты (которые вполне самостоятельны и не могут же почитаться один развитием другого), не только изменяются и совершенствуются \emph{повозрастно}\footnote
{Курсив наш.}), но еще и разнообразятся по культурно\-/историческим типам. Поэтому, собственно говоря, только внутри одного и того же типа, или, как говорится, цивилизации\---и можно отличать те формы исторического движения, которые обозначаются словами: древняя, средняя и новая история. Это деление есть только подчиненное, главное же должно состоять в отличении культурно\-/исторических типов, так сказать, самостоятельных, своеобразных планов религиозного, социального, бытового, промышленного, политического, научного, художественного, одним словом исторического развития>>\footnote{Данилевский. Россия и Европа, стр.\=,88.}).

Борьба обоих идеологов национализма направлена главным образом против единого человечества, как субъекта исторического прогресса и против всемирно\-/исторической точки зрения, утверждающей преемственность культурных приобретений и существование общечеловеческих форм общественного устройства, художественного творчества, научной и философской мысли.

<<Германская философия,\---говорит Данилевский,\---с презрением устраняя всё имевшее сколько\-/нибудь характер случайности и относительности, схватилась бороться с самим абсолютным и, казалось, одолела его. Так же точно социализм думал найти общие формы общественного быта, в своем роде также абсолютные, могущие осчастливить все человечество, без различия времени, места или племени. При таком направлении умов понятно было увлечение общечеловеческим>>\footnote{Данилевский. Россия и Европа, стр.\=,120\--121.}). Данилевский был знаком с так назыв. французским, т.\=,е. утопическим социализмом, который все несовершенства общественного устройства считал плодом невежества, а не необходимым следствием экономических условий исторического развития. Но современный научный социализм, с которым Шпенглер, хотя и плохо, но все же знаком, выводит сходство различных народов из одинаковых условий экономического быта и его развития. Марксизм дает вместе с тем и научное объяснение возможности <<передачи>> начал культуры одним народом другому, поскольку их общественное сознание определяется одинаковым или сходным общественным бытием. Но метафизика Шпенглера и Данилевского, разумеется, не может удовлетвориться таким эмпирическим объяснением исторических явлений.

<<Общечеловеческой цивилизации,\---говорит Данилевский,\---не существует и не может существовать, потому что это было бы только невозможная и вовсе нежелательная неполнота>>. Существуют только цивилизации определенных культурно\-/исторических типов, независимых друг от друга, при чем начала цивилизации одного культурно\-/исторического типа не передаются народам другого типа. Каждый тип вырабатывает её для себя и из себя. <<Прогресс,\---говорит Данилевский,\---состоит не в том, чтобы идти все в одном направлении (в таком случав он скоро прекратился бы), а в том, чтобы исходить всё поле, составляющее поприще исторической деятельности человечества, во всех направлениях. Поэтому ни одна цивилизация не может гордиться тем, чтобы она представляла высшую точку развития, в сравнении с её предшественницами или современницами, во всех сторонах её развития>>. <<Дабы поступательное движение вообще не прекратилось в жизни всего человечества, необходимо, чтобы, дойдя в одном направлении до известной степени совершенства, началось оно с новой точки исхода и шло по другому пути, т.\=,е. надо, чтобы вступили на поприще деятельности другие психические особенности, другой склад ума, чувств и воли, которыми обладают только народы другого культурно\-/исторического типа>> \footnote{Данилевский, Россия и Европа, стр.\=,114\--115.}). Таким образом, формулировка культурно\-/исторических типов у Данилевского почти тожественна с определением Шпенглера. И, стало быть, великое открытие, которое Шпенглер приписывает исключительно своей гениальности, сделано задолго до него. Любопытно отметить в этой связи, что и Данилевский, подобно Шпенглеру, сравнивает это свое <<открытие>> с Коперниковским переворотом <<в астрономии>>. <<Как только ложное понятие о центральности земли было заменено естественною системою Коперника, т.\=,е., каждое небесное тело поставлено и в умах астрономов на подобающее ему место, сейчас же открылась возможность определять относительное расстояние этих тел от солнца>>. Историки находятся в том же положении, как и астрономы. <<Эти последние могут определять, со всею желаемою точностью, орбиты планет, которые во всех точках подлежат их исследованиям\---могут даже приблизительно определять пути комет, которые подлежат исследованиям только в некоторой их части; но что могут они сказать о движении всей солнечной системы, кроме того разве, что и она движется, и кроме некоторых догадок о направлении этого движения? Итак, естественная система истории должна заключаться в различении культурно\-/исторических типов развития, как главного основания её делений, от степеней развития, по которым только эти типы, а не совокупность исторических явлений могут подразделяться>> \footnote{Данилевский, Россия и Европа, стр.\=,91.}).

Установив далее определенные признаки самобытного культурноисторического типа, Данилевский в полном согласии с Шпенглером приходит к тому заключению, что Европа представляет собою культурно\-/историческую единицу, объемлющую собою германо\-/романский мир. Предвидя возможное возражение, что понятие общеевропейской цивилизации однозначно с преодолением национальной ограниченности и признанием общечеловеческого, Данилевский спешит опровергнуть этот аргумент следующим рассуждением: <<Здесь,\---говорит он,\--- не принималось во внимание того, что Франция, Англия, Германия были только единицами политическими, а культурной единицей всегда была Европа в целом, что, следовательно, никакого прорвания национальной ограниченности не было и быть не могло, что германско\-/романская цивилизация, как была всегда принадлежностью всего племени, так и осталась ею>> \footnote{Данилевский, Россия и Европа, стр.\=,120}). Это опять\-/таки вполне в духе Шпенглера, который политические организмы\---Германия, Англия, Франция\---<<диалектически>> объединяет в высшем понятии фаустовского, т.\=,е. западно\-/европейского или германо\-/романского культурно\-/исторического типа.

Ход развития культурно\-/исторических типов Данилевским уподобляется законам развития растительного и животного организма который рождается, достигает зрелости, затем возмужалости и, наконец, старости, при чем период роста бывает неопределенно продолжителен, но период цветения и плодоношения\---относительно короток и истощает раз навсегда их жизненную силу. В этнографический, т.\=,е. первоначальный, подготовительный период, начинающийся с самого момента выделения культурно\-/исторического племени от сродственных с ним племен и измеряемый тысячелетиями, собирания запаса сил для будущей сознательной деятельности, закладываются те особенности в складе ума, чувства и воли, которые составляют всю оригинальность племени, налагают на него печать особого типа общечеловеческого развития и дают ему способность к самобытной деятельности, без чего племя было бы общим местом, бесполезным, лишним, напрасным, историческим плеоназмом в ряду других племен человеческих. За этнографическим периодом следует период государственный, <<в который народы приготовляют, так сказать, место для своей деятельности, строят своё государство и ограждают свою политическую независимость, без которой, как мы видели, цивилизация ни начаться, ни развиться, ни укрепиться не может>>. Наконец, за государственным периодом наступает \emph{цивилизационный.} <<Если этнографический период есть время собирания, время заготовления запаса для будущей деятельности, то период цивилизации есть время растраты\---растраты полезной, благотворной, составляющей цель самого собирания, но все\-/таки растраты; и как бы ни был богат запас сил, он не может, наконец, не оскудеть и не истощиться\---тем более, что во время возбужденной деятельности, порождающей цивилизацию и порождаемой ею, живется скоро. Каждая особенность в направлении, образовавшаяся в течение этнографического периода, проявляясь в период цивилизации, должна непременно достигнуть своего предела, далее которого идти уже нельзя, или, по крайней мере, такого, откуда дальнейшее поступательное движение становится уже медленным и ограничивается одними частными приобретениями и усовершенствованиями. Тогда происходит застой в жизни, прогресс останавливается; ибо бесконечное развитие, бесконечный прогресс в одном и том же направлении (а еще более во всех направлениях разом), есть очевидная невозможность>> \footnote{Данилевский, стр.\=,113.}). Мы видели, что Шпенглер различает те же три периода: этнографический, культурный и цивилизационный.

По вопросу об определении возраста культурно\-/исторического типа, продолжительности его жизни, как и возможности посредством аналогии исторического предвидения, Данилевский опять - таки обнаруживает удивительную солидарность с Шпенглером. <<Рассматривая историю отдельного культурного типа,\---говорит Данилевский,\--- если цикл его развития вполне принадлежит прошедшему, мы точно и безошибочно можем определить возраст этого развития,\---можем сказать: здесь оканчивается его детство, его юность, его зрелый возраст, здесь начинается его старость, здесь его дряхлость, или, что то же самое, разделить его историю на древнейшую, древнюю, среднюю, новую, новейшую и т.\=,д. Мы можем \emph{сделать это с некоторым вероятием, при помощи аналогии, даже и для таких культурных типов, которые ещё не окончили своего поприща>>} \footnote{Данилевский, стр.\=,90.}). Высказанное в этих строках положение почти дословно находим у Шпенглера, который, однако, уверяет, что никто никогда к этой <<великой>> истине даже и не приближался.

Знаменитый \emph{морфологический принцип,} составляющий предмет особой гордости Шпенглера, фигурирует у Данилевского даже под тем же названием и развит со всей определенностью. Науки о духе,\---говорит Данилевский,\---не существует безотносительно к определенным формам. Они имеют <<своим предметом лишь видоизменения материальных и духовных сил и законов\---под влиянием орфологического принципа>>, т.\=,е. в соединении с известными формами. Эти науки не вырабатывают общих теорий, а отыскивают лишь частные законы. Физиология, анатомия, науки филологические, исторические и общественные не могут быть науками теоретическими, а лишь науками \emph{сравнительными.} <<Общественные явления не подлежат никаким особого рода силам, следовательно, и не управляются никакими особыми законами, кроме общих духовных законов. Эти законы действуют особым образом под влиянием морфологического начала образования обществ; но так как эти начала для разных обществ различны, то и возможно только не теоретическое, а лишь сравнительное обществословие и части его: политика, политическая экономия и т.\=,д.>> \footnote{Данилевский, стр.\=,169\==170.}). И далее: <<Теоретическая политика или экономия так же невозможна, как невозможна теоретическая физиология или анатомия. Все эти науки и вообще все науки, за исключением трех вышеупомянутых (т.\=,е. химии, физики и психологии А.\=,Д.), могут быть только сравнительными. Следовательно, за неимением теоретической основы,\---каких\-/либо особенного рода самобытных, не производных экономических или политических сил и законов,\---все явления общественного мира суть явления \emph{национальные} (курсив наш А.\=,Д.) и, как таковые, только и могут быть изучаемы и рассматриваемы>> \footnote{Данилевский, стр.\=,170.}). Итак, все науки национальны, но наиболее национальным характером отличаются науки общественные, так что говорить об общей теории устройства гражданских и политических обществ нелепо, ибо общественные науки народны по самому своему объекту\dots

Необходимо далее указать, что даже характеристика психического строя культурных типов у Данилевского и Шпенглера почти тождественна. Так, напр., Данилевский, как и Шпенглер, противополагает \emph{геометрический} метод греков \emph{аналитическому методу} германо\-/романского мира и алгебраическому индийцев, при чем это различие в методах объясняется Данилевским различием психических особенностей соответствующих типов. Геометрическая метода требует, чтобы <<геометрическая фигура, свойства которой исследуются, непрестанно представлялась воображению с полною отчетливостью; воображение греков отличалось \emph{пластичностью представлений} (курсив наш А.\=,Д.); отношение предметов и понятий их не удовлетворяло, так как им необходимо было живое, образное представление самих предметов и тел. Что касается индийцев (Шпенглер прибавил бы и \emph{арабов} А.\=,Д.)\---этих изобретателей \emph{алгебры,} то отличительное свойство их психического склада\---богатство \emph{фантазии}>>. <<Воображение индийцев сочетает и нагромождает самые странные фантастические, и вместе с тем и самые неясные, неотчетливые>>. Германо\-/романский тип отличается особыми психическими свойствами, результатом которых является особая своеобразная культура. И если взять самое абстрактное её выражение\---математику, то мы видим здесь господство \emph{анализа.} <<При методе аналитической, составив из рассмотрения фигуры уравнение, которое связывало бы между собою некоторые существенные свойства фигуры, подвергают это уравнение процессу диалектического развития, совершенно оставляя в стороне представление о самой фигуре. Из этого диалектического развития, если оно произведено правильно, вытекают сами собою выводы, к которым могут подать повод свойства фигуры>> \footnote{Данилевский, стр.\=,144\--145.}).

Таким образом, между Шпенглером и Данилевским существует полное единодушие по всем вопросам. Мы считаем лишним и бесполезным делом подвергнуть критике принцип <<народности>> в науке, искусстве и политике. Уж слишком несостоятельна эта точка зрения. Разве можно согласиться с мыслью о существовании славянской и германо\-/романской политической экономии или физиологии?

Разве можно, в самом деле, сомневаться в том, что <<цивилизации>> передаваемы, если для этого существуют определенные общественные условия? Разве можно брать в серьез мнение Данилевского и Шпенглера, что анализ, скажем, является исключительным продуктом <<фаустовского>> человека и недоступен греку, или что алгебра\---выражение психического склада индийцев и арабов и чужда другим культурно психическим типам? До какой степени это неверно, доказывает факт, установленный историками математики, что арабы свои сведения из алгебры заимствовали отчасти от индусов, отчасти от греков, что арабы в известном отношении даже сделали шаг назад, отказавшись от символических обозначений: их алгебра, как говорит один математик, чисто риторическая. <<Геометрическим умом>> греков было положено начало научной арифметике и алгебре. Мало того, новейшими исследованиями установлено, что учение о бесконечно\-/малых было хорошо известно великому материалисту Демокриту, что он является его основоположником, ибо <<атом>> есть для него не что иное, как <<дифференциал>>. Алгебра александрийского математика Диофанта (IV\=,ст. по Р.\=,X.) господствовала в Европе до середины XVII\=,ст., из чего следует, что западно\-/европейский человек воспитывался на греческой и арабско\-/индусской алгебре. Ну, а современное культурное человечество не воспитывается разве на <<эвклидовых началах>> и на индусской или арабской алгебре? Современная математика, как и вся современная наука и философия, представляет собою не что иное, как дальнейшее \emph{развитие} античной научной и философской мысли, обусловливаемое и определяемое, разумеется, конкретными общественными условиями. Стало быть, существует и эволюция и <<передача>> начал культуры. Из того факта, что научная и философская мысль не застыла на определенной ступени развития, что мы в сравнении с античным миром сделали во всех отношениях значительный шаг вперед, словом, из факта развития делается нелепый вывод о \emph{застойности и неизменности} начал кульуры. Это достигается тем, что \emph{ступени развития} превращаются в самостоятельные и самобытные \emph{типы бытия,} между которыми устанавливается метафизическая противоположность и абсолютная <<прерывность>>. <<Глубокомысленная>> метафизика Шпенглера\-/Данилевского ведет, таким образом, неизбежно к отрицанию эволюции и человеческого прогресса, к крушению науки и всякого объективного знания. Но наши идеологи национализма хорошо чувствуют, в каком месте <<башмак жмет>>. Оба с одинаковым ожесточением нападают на дарвинизм и социализм (даже в этом отношении между ними полное сходство), хорошо сознавая, что идея эволюции и научного объективизма составляет серьезную опасность для их идеологии. Шпенглер прекрасно отдает себе отчет в том. что единственный серьезный враг национализма и теории замкнутых культурных типов\---это марксизм и марксовский интернационализм, стремящийся к утверждению <<всеобщего братства>> или единства человечества на почве социалистического переустройства мира. В борьбе с марксизмом Шпенглер выдвинул свою философию истории и идею <<социалистической монархии>>, т.\=,е. идею истинно\-/прусского социализма и имперализма\---в надежде, что немецкие рабочие дадут себя увлечь на путь шовинизма. Не подлежит сомнению, что Шпенглера постигнет жестокое разочрование.

\begin{flushright}
 \textbf{А.\=,Деборин.}\hspace*{2em}
\end{flushright}
\begin{center}
 \vspace{2em}
 \rule{7em}{1pt}
 \vspace{2em}
\end{center}

\section*{Наши российские шпенглеристы}
\addcontentsline{toc}{section}{3. В.\=,Ваганян\---Наши российские шпенглеристы}
\label{sec:3}

\begin{center}
\noindent(<<Освальд Шпенглер и закат Европы>>\---сборн. статей Н.\=,Бердеява, Я.\=,Букшпана, Ф.\=,Степуна и С.\=,Франка.\---Кн\=/во <<Берег>>. Москва, 1922\=,г. стр.\=,96).
\end{center}

<<Такие книги, как книга Шпенглера, не могут не волновать нас. Такие книги нам ближе, чем европейским людям. \emph{Это\---нашего стиля книга}>> \footnote{См. стр.\=,72}).

Это слова Н.\=,Бердяева, так резюмирует он свою статью, но это основной мотив всего сборника, так на разный лад и по разному хвалят Шпенглера все его участники.

Не во всём четыре автора согласны между собой, по разному они оценивают эту модную философию: одни восторженны, другие снисходительны, но и восторг, и снисходительная улыбка вызваны одной и той же причиной\---сознанием тесной и глубокой связи ее с <<русской>> философией.

Все четыре автора безусловно сходятся на этом: если кого и можно назвать родоначальником модной ныне философии Шпенглера, так это славянофилов: Данилевского, Достоевского, К.\=,Леонтьева, Н.\=,Бердяева.

Что\-/ж. Охотно согласимся с этим. Философия Шпенглера есть весьма во многом повторение в более или менее талантливой, в более или менее острой форме самобытной российской философии.

Успех ее в столь острой постановке, какая дана Шпенглером, обусловлен состоянием умов в нынешней буржуазной Европе, потерпевшей невероятную катастрофу в мировой войне, потерявшая в ней и устойчивость, и веру в свою силу и свои идеалы. Почва пошатнулась под ногами, раскаты революционных бурь до осязательности реально висят в воздухе; тяжелое предчувствие гибели культуры, кошмар варваризации и уничтожение всех достижений нынешней культуры есть ни что иное, как слепое, не осознанное предчувствие мировой революции и гибели культуры капиталистической, культуры банкиров, лавочников и рантье.

Несомненно этот лавочник, банкир и рантье\-/мещанин и создали шумный успех философии <<славянофила от Пруссии>>\---Шпенглера.

Это столь очевидная истина, что признается даже Ф.\=,Степуном (стр.\=,32). Однако, из этого признания господин Степун делает весьма своеобразный, можно сказать, смелый вывод: <<успех книги Шпенглера,\--- говорит он,\--- \emph{означает благостное пробуждение лучших (!) людей Европы} к каким\-/то новым тревожным чувствам, к чувству хрупкости человеческого бытия и \glqq распавшейся цепи времени\grqq, к чувству недоверия, к разуму жизни, к логике культуры, к обещаниям заносчивой цивилизации, к чувству \emph{вулканической природы всякой исторической почвы}>>. Это недурно сказано. Только почему это <<лучшие люди>>\---все эти западно\-/европейские лавочники\---трудно сообразить. Ну, да, оно и понятно, тот класс, в котором это всемирное потрясение вызвало волю к жизни, к победе и к <<вулканическим>> действиям может ли быть понятым г.\=,Степуном? Для него и для его товарищей этот новый класс и его борьба и выражают всего отчетливей кризис культуры, ее упадок и признак ее разложения.

Вот послушайте г.\=,Франка, он находит бесспорным факт умирания <<\emph{в каком}\-/то смысле>> <<Запада>> и <<западной культуры>>. <<Весь вопрос в том: \emph{в каком именно} смысле или, точнее, \emph{какая именно} \glqqзападная культура\grqq умирает (50\=,стр). Вы заинтригованы такой постановкой вопроса и он, после некоторой возни с философией истории Николая Кузанского, отвечает на этот вопрос: \glqqЭто есть конец того, что зовется \glqqновой историей\grqq. \glqqВеличайший объек- тивный трагизм переживаемой нами эпохи состоит в том,\---говорит С.\=,Франк,\---что поверхность исторической жизни залита бушующими волнами движения, руководимого духовно отмирающими силами ренессанса, а в глубинах жизни, еще совершенно бездейственно и уединенно назревают потоки нового движения, которому, быть может, суждено сотворить новую культуру, искупив основное грехопадение ренессанса>>. Сколько напущено туману! А в сущности смысл этой фразы весьма несложен.

Кто <<руководит>> движением, залившим бушующими волнами на <<поверхность исторической жизни>>?

Слепому видно\---рабочий класс. Какие силы вызвали его\---это движение? Противоречия капиталистического общества.

Что есть <<духовно отмирающие силы ренессанса>>?

Пролетариат и буржуазия, ведущие исторический поединок на глазах у всех.

Что С.\=,Франк считает пролетариат <<духовно отмирающей силой ренессанса>>\---в этом ничего ни неожиданного, ни странного нет, гораздо более удивительно то, что он, на первый взгляд, склонен и буржуазию похоронить заодно с ненавистным ему рабочим классом; или не даром прошли для господина С.\=,Франка четыре года революции?

Однако это так кажется лишь на первый взгляд. На самом деле господин С.\=,Франк и не думает похоронить буржуазию. Он видит печать смерти на лбу пролетариата с его социализмом, коммунизмом, интернационализмом; он их и считает подлинными <<силами ренессанса>>.

Но если замысловатый С.\=,Франк говорит это с <<ужимкой>>, то болтливый Н.\=,Бердяев несколькими страницами Ниже говорит это совершенно открыто.

Но прежде покончим с господином Франком.

Он кроме того, что вещает, подобно пифии, смерть руководителям <<бушующими волнами движения>>, указывает и на спасителя, которому, <<быть может, суждено сотворить новую культуру>>? Что это за спасительная сила.

Та самая, которая явилась, как реакция против материализма XVIII столетия\---романтизм и идеализм. Они <<подземной струёй>> дошли до наших дней и <<образуют теперь \emph{самую глубокую и духовно влиятельную} силу внутреннего идейного творчества>>. Даже социализм, как идейное движение, одним боком примыкает к этим двум истокам идейного творчества: через Сен\=/Симона с романтизмом, а через Гегеля с немецким идеализмом.

Так вот та сила, которая придет создавать новую культуру на развалинах старой, отличительной чертой которой были наука (рационализм) и материализм. Это очень похоже на Шпенглера и немного не похоже. <<Ученая книга Шпенглера\---явный вызов науке>>,\--- говорит г.\=,Степун;\---не меньший вызов науке, когда С.\=,Франк пишет: <<Человечество вдалеке от шума исторических событий накопляет силы и духовные навыки для великого дела, начатого Данте и Николаем Кузанским и неудавшегося благодаря \emph{роковой исторической ошибке или слабости их преемников>>.} Эта <<роковая историческая ошибка>> прямо великолепна. Она так похожа на истинно российский <<шпенглеризм>>!

Но если оставим в стороне юродства господ Франков и попытаемся понять их мысль, то она весьма не сложна, она вся сводится к поговорке <<и на нашей улице будет праздник>>.

Если бы г.\=,Франк не писал, а говорил, да говорил бы сам с собой без свидетелей, не жеманясь и не затемняя речь, то получилась бы примерно такая речь: <<После Великой французской революции вместо материализма (\glqq в XVIII веке философия окончательно обездушивает мир и жизнь\grqq) расцвёл романтизм (Байрон) и немецкий идеализм; когда кончится нынешняя революция\---человечество, которое стоит вдали от шума исторических событий (а вы не думаете, господин Франк, что они были \emph{насильственно} и навсегда отдалены от шума исторических событий? Я думаю это очень похоже на правду), выйдет на сцену и начнет строить новых богов (российский идеализм всегда с этого начинает свою карьеру, хотя кончает тоже на одном и том же роковом месте), устраивать оргии и возвешать культ женского тела (российский романтизм тоже всегда начинает свой танец от этой печки) >>.

Это верно.

В таком разговоре, если нет правды, то есть возможность её. Дело вот в чем.

Вслед за каждой революционной бурей наступала более или менее длительная эпоха реакции. Чем глубже захватывала революция, чем сильнее бушевала она, чем больше масс охватывала, вовлекала в движение, тем сильнее, тем длительней, тем черней бывала реакция.

В дни революции, особенно после 1848\=,г., развертывалась вся творческая сила трудящихся и угнетенных, в дни реакций махровым цветком распускался мистицизм, идеализм, романтизм,\---словом, выходили на свет все те, кто <<ютился вдали от шума исторических событий>>. И само собой разумеется, наша революция в этом отношении не составит исключения, буде она уступит место реакции.

Но в том\-/то и весь вопрос\---уступит ли? Наступит ли \emph{та тишь,} которая нужна людям ожидающим <<вдалеке от шума исторических событий>>?

Огромное общественное значение реферируемого сборника в том именно и заключается, что авторы его видят в Шпенглере признак наступления этой \emph{тиши.} Их объединяет прежде всего уверенность в том, что наступает эта эпоха исканий бога, эпоха романтизма и идеализма.

Степун думает, что книга Шпенглера\---предзнаменование какого\-/то нового углубления религиозной мистической жизни Европы (33) и с этим согласны все авторы.

Но сборник этот имеет и другой симптоматический смысл и значение: он указывает на то, что <<человечество>>, которое ютится на время бури в отдаленном уголке большого корабля, бережно, подобно маньяку, хранит у себя старую идеологию национализма, носится с ним и готовится сделать его одним из составных частей грядущей на смену умирающему <<ренессансу>> культуры.

<<Культура\--- национальна, цивилизация\--- интернациональна>> вслед за Шпенглером повторяет Н.\=,Бердяев (последний уверяет, что Шпенглер повторяет за ним\---для нас с вами читатель это не существенно), и если цивилизация должна погибнуть во имя новой культуры, то должен погибнуть и интернационализм во имя национализма. <<Цивилизация по существу своему проникнута духовным мещанством, духовной буржуазностью. Капитализм и социализм \emph{совершенно одинаково} заражены этим духом>>. <<Культура имеет религиозную основу, в ней есть священная символика. Цивилизация же есть царство от мира сего. Она есть торжество \glqqбуржуазного\grqq духа, духовной \glqqбуржуазности\grqq. И \emph{совершенно безразлично,} будет ли цивилизация капиталистической или социалистической, она одинакова\---безбожная, мещанская цивилизация>>. И если цивилизация погибнет и придёт некая новая культура, то погибнет безбожие и восторжествует религия. Н.\=,Бердяев всех типичней из всех участников сборника, как шпенглерист. Шпенглер\---прусский националист; он находит, что миру Пруссия призвана сообщать нечто новое что ей принадлежит первое слово завтра, а не какой\-/либо другой стране; Н.\=,Бердяев также думает только\dots о России; <<Что бы ни было с нами, мы неизбежно должны выйти в мировую ширь. Россия\---посредница между Востоком и Западом. У ней сталкиваются два потока всемирной истории\--- восточный и западный. В России скрыта тайна, которую мы сами не можем вполне разгадать. Но тайна эта связана с разрешением какой\-/то темы всемирной истории: Час наш еще не настал. Он связан будет с кризисом европейской культуры>>.

И, как из\-/под 700 страниц <<Заката Европы>> Шпенглера, остроумного, местами быть может весьма талантливого, выглядывают большие уши прусского национализма, жаждущего реванша, грезящего о новом всемирном покорении\---\emph{прусского мессианизма,} так из\-/под всего писания нашей российской интеллигенции, ютившейся до сих пор <<вдали от шума исторических событий>>, выглядывает все тот же старый заскорузлый национализм, слепой, ничему не научившийся.

Эта идея русского мессианизма прикрывала тягу русских империалистов к Дарданеллам, как идея прусского культуртрегерства\---прямой разбой Германии.

От этого сборника до новой <<Великой России>> так же близко, как близок был переход от <<Вех>> к <<Великой России>> кануна войны.

К чему же свелись все разговоры о гибели цивилизации, <<Закат Европы?>>

К весьма грубому, весьма неискусно прикрытому, апологию национализма и реакции.

В чем спасение человечества?

В возврате к былым, довоенным идеям и идеалам буржуазии?

Нет, в том\-/то и дело, что нет.

В преодолении довоенных форм капитализма дальнейшим развитием империализма, в обострённом развитии чувства национализма. И с этой целью\---в преодолении социализма\---мистицизмом.

Это в конечном итоге очень прозаично, очень голо, но, думается, всякому внимательному читателю очевидно.

Мы тоже думаем, что <<Запад>> переживает свой закат, свои сумерки.

Но какой запад и какие сумерки. Капитализм дошел уже до той роковой для нее степени развития, когда <<материальные производительные силы общества>> впали <<в противоречие с существующими производственными отношениями\dots с имущественными отношениями>>. Ему ничего не остается, как в предсмертной агонии отождествлять свою гибель с гибелью культуры, человечества\---это не впервые; не один господствующий класс погибал за долгие годы человеческой истории, проделывая при этом буквально то же, что сегодня, буржуазия, отождествляя свою пбель с гибелью культуры, мира, приходом антихриста или еще что либо в этом роде.

Пролетариат отличается от всех до него существовавших революционных классов тем, и прежде всего тем, что он осознал законы исторической необходимости и, следовательно, он свободен по отношению к ним и действует методически и последовательно в направлении развития исторической необходимости. Ему не страшны никакие антихристы, как не страшен сегоднешний обвал огромного здания капитализма. Он не приходит в отчаяние от гибели старого мира и культуры лавочников. Из безформенной груды развалин старого мира эксплоатации и ветхой культуры он создаст новую культуру и свой мир труда и свободы.

\begin{flushright}
 \textbf{В.\=,Ваганян.}\hspace*{2em}
\end{flushright}
\begin{center}
 \vspace{2em}
 \rule{7em}{1pt}
 \vspace{2em}
\end{center}
