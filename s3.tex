\section*{Обломки старой России}
\addcontentsline{toc}{section}{4. Е.\=,Преображенский\---Обломки старой России}
\label{sec:4}

В Петрограде издаются два журнальчика: <<Летопись дома литераторов>> и <<Вестник литературы>>, на страницах которых инвалиды от интеллигенции устроили маленький затхлый чуланчик для проветривания остатков своего литературного добра. Среди участников мелькают знакомые имена, в том числе небезызвестный А.\=,С.\=,Изгоев.

О статье Изгоева и его докладе, сделанном о <<Смене вех>> и помещенном в кратком изложении, я хотел бы сказать пару слов.

После разгрома первой русской резолюции 1905\--6 годов определенная часть русской интеллигенции покинула народнические и демократические позиции и, поплевав основательно <<об это место>>, стала идейно выравниваться под буржуазию, которая в свою очередь на основе капиталистического развития, достигнутого к тому времени Россией, стала выравниваться в области экономической под буржуазию Запада. Сборник <<Вехи>> был декларацией этой группы, при чем переход в лагерь буржуазии, как это всегда бывает в таких случаях, был замаскирован всякими идеологическими вывертами, а в данном случае густейшим туманом всяческой мистики, разговорами о необходимости создания национальной культуры, национальной традиции, и т.\=,д. и т.\=,д. В числе видных лидеров этой ренегатской группы был А.\=,С.\=,Изгоев, специализировавшийся на вопросе об интеллигенции. Поскольку тут дело шло о нашей народнической интеллигенции, то основную плевательную роль по отношению к ней выполнял именно он.

С тех пор утекло много воды. Российская буржуазия, на которую ориентировались <<веховцы>>, потерпела кораблекрушение в классовой войне с пролетариатом. Победа пролетариата оказалась настолько прочной и решающей, удельный социальный вес его вместе с крестьянством так подавляюще велик, революция так непобедима, что часть антисоветской интеллигенции, в том число часть бывших <<веховцев>>, проделала такой же самый переход от буржуазии к советской демократии, какой прежние веховцы проделали от народнической демократии к буржуазии. А.\=,С.\=,Изгоев остался верен <<старому барину>>. Подобно старику лакею Фирса из Чеховского <<Вишневого сада>>, он вместе со своими единомышленниками из старых публицистов, поэтов и литераторов продолжает ревностно убирать полы и окна старого дома, прислуживая буржуазии <<по идейной части>>. На склоне лет старуха буржуазия раскаивается в грехах своей буйной молодости и после атеизма и материализма склонна к религии и мистицизму. И гг.\=,Изгоевы, Булгаковы, Бердяевы говорят ей об <<абсолютных, религиозных критериях (статья Изгоева \glqq Власть личности\grqq), о том, что человек всегда в конце\-/концов останется наедине со своим Богом, со своей совестью и перед ним принужден будет держать ответ>> (доклад Изгоева о <<Смене вех>>). Но Чеховский Фирса естественней, проще, правдивей и пожалуй симпатичней в своей простоте идеологических лакеев буржуазии из интеллигенции. Он благоговеет перед своим господином, дрожит от счастья, целуя барскую руку, и гордится своей службой перед всеми <<недотепами>>, которые не умеют обращаться с <<господами>>. Наоборот, интеллигентный лакей, стирая идейное белье Рябушинскому, будет говорить о том, что служит абсолютной идее чистоты вообще и, украшая поэтическими розами колесницу эксплоататоров, будет это делать во имя вечной и абсолютной красоты. И в своей статье <<Власть и личность>> А.\=,С.\=,Изгоев не просто целует барскую руку без затей, а считает нужным покривляться, выдвигая проплесневшую идею о независимой, внеклассовой интеллигенции, которая не должна преклоняться ни перед какой государственной властью. Пока, в нашей конкретной действительности это означает призыв не преклоняться перед реально существующей Советской властью. А так как Изгоев не анархист, а в природе теперь существует кроме пролетарской лишь буржуазная или буржуазно\-/помещичья власть, к которой Изгоев со времени <<Вех>> не относился отрицательно, то каждому понятно, о чём в сущности идет речь.

Говорить же, что никогда и нигде интеллигенция не представляла из себя самостоятельной классовой силы и всегда тяготела и идейно обслуживала тот или иной из основных классов современного общества, вряд ли стоит. Но само свойство профессии интеллигенции и интерес общественного класса, которому она служит, требуют такого самоббмана и обмана масс.

Что касается до выступления старого веховца против нововеховцев, то, поскольку Изгоев считает напрасными, попытки нововеховцев видеть наш коммунизм и интернационализм за обыкновенный великодержавный и великорусский национализм, он прав \emph{с нашей точки} Зрения и когда он упрекает их за переход на сторону Советской власти\---он прав \emph{с своей точки зрения,} т.\=,е. с точки зрения контр революционной буржуазии. Но переход нововеховцев на сторону революции нужно рассматривать по объективным результатам этого шага, не перенося центр тяжести вопроса на мотивы и аргументацию, которая при этом пускается в ход. При всех исторически значительных актах в социальной борьбе имеют значение реальные результаты, а не слова, которые при этом говорятся, и не настроения действующих лиц. По каким субъективным мотивам Ключников, Устрялов и др. пришли к оправданию революции, имеет второстепенное значение, как не имеет большого значения и то, что они, помогая Советской власти, думают помочь новой буржуазии и великодержавному национализму. В <<Былом и Думах>> Герцен в одном из своих блестящих сравнений говорил о том, что <<полип>>, умирая, служит прогрессу <<рифа>>. Так и часть интеллигенции, переходя на сторону революции и умирая, как белогвардейская сила, служит прогрессу Советского рифа. Об этот риф разобьётся корабль буржуазного господства, а вместе с тем разобьются одинаково и иллюзии тех, которые оседают на этот риф в надежде укротить революцию, так и надежды тех, которые склоняют голову перед поповским паникадилом на борту тонущего корабля.


\begin{flushright}
 \textbf{Е.\=,Преображенский.}\hspace*{2em}
\end{flushright}
