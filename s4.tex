\newpage
\noindent\textbf{Г.\=,В.\=,Плеханов.}
\section*{Огюстен Тьерри и материалистическое понимание истории\footnote{Статья появляется на русской языке впервые. Была помещена в журнале <<Le Devenir Sozial>> за ноябрь 1895 г. № 8.}}
\addcontentsline{toc}{section}{5. Г.\=,В.\=,Плеханов\---Огюстен Тьерри и материалистическое понимание истории}
\label{sec:5}

Огюстен Тьерри прнадлежит к замечательной группе тех известных ученых, которые в эпоху Реставрации возобновили во Франции исторические исследования. В этой группе не было ни учителя, ни учеников. Тем не менее она образует настоящую школу, основные концепции которой весьма полезно рассмотреть.

Шатобриан \footnote{Шатобриан. Исторические Очерки. Предисловие.}) обозначил эту школу именем школы политической. Это не точно. В самом деле,\---философы XVIII века твердо убежденные в том, что положение народа \emph{всецело определяется законодательством} умели связывать <<законодательство>> только с преднамеренным действием законодателя.\footnote{См. между тысячами других примеров, <<Наблюдения>> Мысли над историей греков и римлян, а также труды Гельвеция и Гольбаха. Религия \emph{Авраама} была повидимому первоначально теизмом измышлённым для того, чтобы преобразовать халдейские суеверия теизм Авраама был извращён Моисеем, которые этим воспользовался для создания иудейских суеверий <<Systemedela Nature>> <<Лондон, вторая часть, стр.\=,186. (Чтобы реформа в Спарте не оказалась лишь временной, он (Ликург) проник так сказать до дна в сердца граждан, и задушил в нём зародыш любви к богатству>>. (Полное собравие соч. Мабли
4\=/й том, стр.\=,20).}) Это и есть точка зрения политическая раг ехеllеnce. Отсюда естественно вытекает, что гражданские законы каждого данного народа обязаны своим происхождением его \emph{политической} конституции, его \emph{правительству.} Философы неустанно повторяли это.

\emph{Для Гизо истинно как раз противоположное.} <<Большая часть писателей, историков и публицистов, говорит он, старалась объяснить данное состояние общества, степень или род его цивилизации политическими учреждениями того общества. Было бы благоразумнее начинать изучение самого общества для того, чтобы узнать и попять его политические учреждения. Прежде, чем стать причиной, учреждения являются следствием, общество создает их прежде, чем начинает изменяться под их влиянием и вместо того чтобы о состоянии народа судить по форме его правительства, надо прежде всего исследовать состояние народа, чтобы судить каково должно быть, каково могло быть его правительство>>\footnote{Гизо. Опыт истории Франции. 10\=/е изд. Париж. 1360\=,г., стр.\=,73 (четвёртый очерк), 4\=/е изд. очерков вышло в 1823\=,г.}.

В этом Минье совершено согласен с Гизо. Для него также политические учреждения являются \emph{следствиями} прежде чем стать \emph{причиной.}

Общественное движение определяется господствующими интересами и это движение определяет и форму правительства. Когда правительство перестаёт соответствовать состоянию народа, оно исчезает. Так феодализм существовал в нуждах людей, еще не существуя фактически; затем он существовал фактически, переставая соответствовать нуждам, отчего прекратилось, наконец, его фактическое существование. Освобождение коммун изменило все внутренние и внешние отношения европейских обществ. Оно дало новое направление политической эволюции Европы. <<Демократия, абсолютная монархия и представительная система явились его результатом; демократия там, где коммунны властвовали самостоятельно, абсолютные монархии там, где они вступали в союз с королями, которых они не могли обуздать\---представительная система там, где вассалы использовали коммуны, чтобы ограничить королевскую власть\footnote{О феодальном строе, об учреждениях Людов. Св. и т.\=,д., Париж 1822\=,г., стр.\=,83.})>>.

Огюстен Тьерри не менее далёк от точки зрения философов XVIII века. <<\emph{Конституции это одежды общества}>>, говорит он. Старая школа уделяла, слишком много внимания генеологии королей. Она не оставляла, места никакой самодеятельности людских масс. <<Если переделяется целый народ и находит себе новое местожительство, то это по словам летописцев и поэтов, некий герой, чтобы прославить свое имя, задумал основать империю; если устанавливаются новые обычаи,\--- это какой\-/либо законодатель измышляет и устанавливает их. Если образовывается город,\---это какой\-/то князь даёт ему существование; и всегда народ, граждане, является материалом для планов одного человека>> \footnote{Об освобождении коммун. Это исследование,\---первый набросок работы по истории третьего сословия был написан в Courrier Francais 13 октября 1820 года.}). Таким образом изложение каждой эпохи становилось расказом о рождении, воспитании, о жизни и смерти законодателя. Эта манера писать историю была естественной для монахов средневековья: монахи писатели питали исключительное предпочтение к тем людям, которые приносили наиболее даров церквам и монастырям. Но этот способ является недостойным для современных историков. То, что нам нужно в настоящее время, это настоящая история страны, \emph{история народа, история граждан.} Эта история представила бы нам одновременно и примеры управления и тот сочувственный интерес, который мы напрасно ищем в авантюрах маленького числа привилегированных лиц, целиком занимающих сцену истории. В наших душах гораздо скорее пробудилась бы привязанность к участи массы людей, которые жили и чувствовали как и мы,\---чем к судьбе вельмож и князей, о которой одной рассказывают нам и которая одна лишь не дает нам полезных уроков. Движение народных масс по пути к свободе и благоденствию нам показалось бы более внушительным, чем шествие завоевателей;\--- а их несчастья более трогательными, чем бедствия лишенных владения королей \footnote{Первая записка об истории Франции напечатана в Courrier Francais 13 июля 1820 года.}).

Таким образом, народ, вся нация должна быть героем истории. Огюстен Тьерри говорит не иначе, как с глухим гневом об этих самых законодателях (завоевателях), к которым беспрестанно взывала историческая школа XVIII века. Это не все. В массе <<граждан>> есть привилегированные и обездоленные угнетатели и угнетаемые. Жизнь этих последних должна привлекать внимание историков. <<Мы их потомки, думаем, что они чего\-/нибудь стоили и что наиболее многочисленная и наиболее забытая часть нации заслуживает того, чтобы воскреснуть в истории. Если дворянство может в прошлом претендовать на высокие воинские подвиги и воинскую славу то\-/есть слава и у простонародья,\---слава мастерства и таланта. Простолюдин дрессировал боевого коня дворянина, он скреплял стальные бляхи его брони; те, кто увеселял замковые празднества музыкой и поэзией, были также из простонародья; наконец, язык, на котором мы сейчас говорим,\---это язык простонародья; оно создало его в то время, когда во дворах и замковых башнях дворянства раздавались грубые гортанные звуки германского наречия>> \footnote{Jbidem}).

Не раз Огюстен Тьерри с гордостью напоминает, что он разночинец, \emph{сын третьего сословия.} И он им остается во всех отношениях. Он становится на сторону этого сословия; его точка зрения,\---точка зрения борьбы простонародья с дворянством, точка зрения \emph{классовой борьбы.} Может быть это удивит не одного читателя. Обычно полагают, что \emph{социалисты марксистской} школы первые ввели эту концепцию в историческую науку\---но это ошибочно. Ода была введена \emph{до Маркса,} она господствовала в той исторической французской школе, которую Шатобриан не точно назвал политической школой, и к которой принадлежал Огюстен Тьерри.

Для Гизо вся история Франции есть борьба, \emph{война} между классами. В продолжение более XIII веков, Франция состояла из двух народов; один народ\---победитель,\---\emph{дворянство;} и другой\--- побежденный\---\emph{третье сословие.} В течение более 13\=/ти столетий, народ\--- побежденный боролся, чтобы стряхнуть иго победителей. Борьба происходила во всех формах и всяким оружием: <<когда в 1789 году представители всей Франции были созваны в одно собрание, эти два народа поспешили возобновить свои старые распри, наконец пришел день покончить с ними \footnote{Гизо. Правительство Франции со времён Реставрации и нынешнее министерство. Париж, 1820\=,г. стр.\=,2\==3.}). Революция изменила взаимоотношение этих двух народов; прежний народ\---побежденный стал победителем, он в свою очередь завоевал Францию. Даже Реставрация была принуждена принять этот совершившийся факт. Хартия объявляла, что этот факт имеет своим источником право, и подписывая Хартию, Людовик XVIII сделался главой новых победителей. Но народ, только что побежденный,\---прежний народ\---победитель,\---не покорился своему поражению. Он продолжает свою старую 13\=/ти вековую борьбу. И в дебатах в Парламенте вопрос ставится как он ставился и прежде, равенство или привилегия, средний класс или аристократия. Мир между ними не возможен. \emph{Примирить их\--- химерический замысел. Привести их к соглашению,\---было бы не менее несбыточной мечтой>>} \footnote{Jbidem стр.\=,108.}).

Здесь нет недостатка ни в ясности, ни в определенности. Но Гизо умел говорить с еще большей ясностью, с ещё большей определенностью. Когда по выходе в свет вышеуказанной работы, его политические враги упрекали его в разжигании гражданской войны, он ответил, что, указав на исторический факт существования борьбы классов, он не сказал ничего нового. <<Я хотел только,\---писал он,\--- вкратце изложить политическую историю Франции. Борьба классов наполняет, или вернее, делает всю эту историю (sicl). Об этом знали и говорили за много веков до революции. Знали и говорили в 1789\=,г., знали и говорили три месяца тому назад. Хотя меня теперь обвиняют в том, что я это сказал, я не думаю, чтобы кто\-/нибудь этого не помнил. Факты не уничтожаются по доброй воле и ради временных удобств министерств и партий. Что сказал бы господин де Буленвилье, если бы, возвратясь в нашу среду он услышал отрицание того, что третье сословие вело войну против дворянства, что оно боролось с ним постоянно за уничтожение его привилегий и за установление равенства с ним? Что сказали бы те многие мужественные буржуа, которые были посланы в Генеральные Штабы для защиты или завоевания прав своего сословия,\---если бы они воскресли, чтобы узнать, что дворянство не вело борьбы с третьим сословием, что оно не поднимало тревоги, видя его рост, что оно не противодействовало всегда его усилению в обществе и укреплению его влияния?>>

Вся эта борьба <<это вовсе не теория, не гипотеза, это сама действительность во всей её простоте>>, и хотя нет ни малейшей заслуги в том, чтобы ее видет, но оспаривать е` это уже почти смешно \footnote{В приложении к двум первым изданиям цитируемой работы (предисловие третьего издания) стр.\=,15.}). Если некоторые сторонники дворянства желали предать её забвению, так это потому, что они больше не считали своё сословие достаточно сильным, чтобы выдержать открытую борьбу и, видя его слабеющим, они старались обмануть средний класс. И Гизо громит их с бурною силой негодующего трибуна. <<Выродившиеся потомки расы, владевшей огромной страной и заставлявшей дрожать великих королей>> восклицает он, <<что же вы отрекаетесь от ваших предков и вашей истории! Чувствуя собственный упадок, вы протестуете против вашего прошлого величия. Так как мы требуем от вас быть отныне лишь равными нам, то вы оспариваете факт, что вы были нашими господами! Я испытал бы стыд, признаюсь в том, если бы мне\--- буржуа\---пришлось бы здесь восстанавливать историю Франции и доказывать противникам конституционного равенства, что они слишком скромны в своих воспоминаниях>> \footnote{Ibid. Стр.\=,8.}).

Будучи художником больше, чем борцом, Огюстен Тьерри никогда не проповедывал классовую борьбу с такой силой и с таким гневом, как это делал Гизо, один из самых замечательных политических борцов французской буржуазии. Но тем не менее он хорошо понимал весь исторический смысл той борьбы, которую среднее сословие вело тогда с дворянством. <<Современное дворянство,\---писал он в 1820\=,г. по поводу работы Вардена о Соединенных Штатах Северной Америки,\---связывает свои претензии с привилегированными людьми XVI столетия. Последние считали себя происходящими от владельцев людей XIII столетия, которые в свою очередь связывали себя с франками. Карла Великого, родословная которых восходила до Сикамбров Хлодвига. Здесь можно оспаривать только естественную преемственность; политическое же происхождение очевидно само собой. Так дадим эту преемственность тем, кто на неё претендует, сами же мы претендуем на преемственность\---противоположную; мы\--- сыновья Третьего Сословия; Третье Сословие вышло из коммун (самоуправляющихся общин); комму вы были убежищами для крепостных. Крепостные были жертвы завоеваний. Итак, от одного вида к другому через промежуток времени в 15 веков, мы приходим к последней разновидности завоевания, которую надлежит стереть. Дай Бог, чтобы это завоевание само отреклось от своих последних следов, и чтобы час боя не должен был пробить. Но без этого, формального отречения не будем надеяться ни на свободу, ни на отдых, не будем надеяться ни на что из того, что делает пребывание в Америке столь счастливым и достойным зависти; плоды, которые приносит эта земля, никогда не станут расти на почве, которая бы оставалась пропитанной остатками захвата>> \footnote{В <<Censour Europeen>> 2\=/ апреля 1820 года.}).

Так или иначе, мирным ли путем, или при помощи <<борьбы>> буржуазия должна уничтожить привилегии дворянства, или как говорил Гизо, и до него еще Сиэйса, побежденный народ должен в свою очередь сделаться завоевателем. Мы могли бы легко найти у Минье и Тьера, страницы, похожие на те, которые мы только что цитировали. Но это бесполезно. Теперь, уже доказано, что когда марксисты говорят о классовой борьбе, они в этом случае следуют только примеру самых выдающихся теориков и историков третьего сословия. Больше того, Гизо нисколько не преувеличивал, говоря, что \emph{представители дворянства} проповедывали эту борьбу также, как представители третьего сословия. В <<Размышлениях над историей Франции>> Огюстена Тьерри, которые предшествует его <<\emph{Рассказам из времён Меровингов}>> читатель найдёт довольно подробный анализ исторических систем до 1789\=,г., дающий ясное представление о том, до какой степени борьба классов, на которые распадалось старое французское общество, влияло на взгляды историков, сторонников того или иного класса. Язык какого\-/нибудь Булэнвиллье или Монлозье часто также отчётлив и энергичен, как язык Гизо или язык агитатора\---марксиста нашего времени.

То, что отличает борьбу классов, проповедываемую французскими историками времён Реставрации от той, которая провозглашается социалистами наших дней\---это, прежде всего, социальное положение того класса, к которому обращаются теоретика социальной войны. Сколько бы историки времён Реставрации ни говорили \emph{о народе, о нации, о массе граждан, о третьем сословии в целом,} все же на самом деле то, что они защищали, это были интересы небольшой части нации, \emph{интересы буржуазии.} Гизо знал это хорошо и говорил об этом без уверток. <<Я знаю\dots, что революция, предоставленная сама себе, свободная от страха, увериная в торжестве, создаст естественно и неизбежно свою собственную аристократию, которая станет во главе общества,\---писал он.\--- Но эта аристократия будет другого рода и будет совсем иначе образована, чем та, обломки которой мы видим>> \footnote{<<Do Couvernement de France>> Etc, стр.\=,108.}). Значит неправда, как это утверждал тот же Гизо, что борьба третьего сословия против дворянства означала борьбу \emph{равенства} против \emph{привилегии.} Дело в сущности говоря, шло о торжестве новых привилегий, привилегий иначе образованных, чем те, с остатками которых боролись Гизо и его друзья. Огюстен Тьерри, вероятно не понимал этого так ясно, как будущий министр Людовика Филиппа. Но и его идеал не превышал торжества \emph{среднего класса.} Вот, например, как он резюмирует историческое дело Великой Французской революции: <<вместо старых сословий, неравных по своим правам и социальному положению (Sic!) классов, образовалось общество однородное; стало 25 миллионов душ, составляющих один единственный класс граждан, живущих при одном законе, одном уставе, одном порядке>> \footnote{<<Размышления над историей Франции>>, предшествующие <<Рассказам из времён Меровингов>>, Париж, 1840\=,г., стр.\=,143.}). Что же оставалось делать?\---Нечего больше, как обеспечить новое общество от нападений сторонников старого режима и защитить завоевания буржуазии от злопамятства дворянства, побеждённого в великой борьбе классов. Правда, даже после 1830 года, когда победа буржуазии стала окончательной, Огюстэн Тьерри, старый учикик и <<приемный сын>> Сен\=/Симона, не находится вполне на стороне удовлетворённых, как Гизо, этот злостный враг всякого движения рабочего класса. Автор <<\emph{Размышлений над историей Франиции}>>, казалось, не вполне осуждал новые социальные и политические тенденции, которые начинают появляться с первых лет царствования Людовика\=/Филиппа. Но он далёк от того, чтобы понять эти тенденции; он желает \emph{социального мира, объединения классов,} он, который при Реставрации проповедывал войну классов. Ибо, ведь, социальный мир при тогдашних условиях не может и не мог бы быть ни чем иным, как, примирением пролетариата с тем ярмом, который налагает на него <<новая аристократия>>\footnote{<<Социальный мир>> сделался также желанием Гизо. Если, после 1848 года он высказывался против Республики, то это объясняется только тем, что Республика не могла обеспечить этот пресловутый мир. <<Само собой очевидно, что демократическая республика, начиная с своих первых действий, близка к тому, чтобы погрузить себя и ввергнуть нас в социальный хаос>>, говорил он в январе 1849 года.}).


Впрочем, справедливо будет вспомнить, что при Реставрации и при Людовике\=/Филиппе, даже теоритики рабочего класса, социалисты и коммунисты не понимали еще того, что пролетариату еще предстоит вести свою социальную войну и одержать свою политическую победу. За очень немногими исключениями они в рабочем вопросе стояли также за более или менее полное объединение классов, а не за их \emph{борьбу. Сен\=/Симон,} которому Огюстэн Тьерри обязан всеми своими историческими идеями, был одним из самых горячих сторонников войны пчёл против трутней. Но пчёлами, для Сен\=/Симона были в этой же самой мере фабрикант и банкир, как и рабочий. И тоже приходится сказать и \emph{о Сен\=/Симонистах. Анфантэн} очень хорошо понимал, что \emph{земельная рента и прибыль с капитала} является продуктом не оплаченного труда <<Собственники,\---говорит он,\---утвердившие за собой землю присваивают себе, с помощью арендной платы часть продуктов, созданных руками трудолюбивых людей. Таков же, в самом деле, и результат отдачи в наём капиталов, а это означает, что работники платят некоторым людям, чтобы последние могли отдыхать и чтобы они оставили в их распоряжении материалы производства>> \footnote{Le Producteur, I Кн. Париж 1825\=,г. Consid\`{e}rations sur la baisse progressive du loyer des objets mobiliers et immobiliers, стр.\=,242\==243.}).

Это хорошо сказано. Но чтоже представляет из себя прибыль предпринимателя, который пользуется взятым в ссуду капиталом? Не является ли она также продуктом эксплоатации рабочих? Нет, отвечает Анфатэн, предприниматель получает свою прибыль благодаря собственному труду. Прибыль и заработная плата зто одно и то же для Анфантэна и в этом именно вопросе он показывает себя совершенно неспособным понять Рикардо, когда английский экономист говорит: there can be no rise in the value of labouar roithout a fall of profits\footnote{<<Рикардо, наивно замечает Анфантэн, всегда подразумевает под прибылью\--- ренту капиталиста. (Анфантэн хочет сказать: ссужающего капитал Г.\=,П.) и говорит, что повышение цены труда уменьшает доход человека, который не работает>> Ibid, стр.\=,545.}. Это превосходно объясняет, почему сен\=/симонисты не хотели и слышать о классовой борьбе. Они были глубоко убеждены, что хозяева и рабочие составляют единый класс, и что их интересы совершенно солидарны. Сен\=/симонисты могли бороться только против <<\glqq класса\grqq\ военвых людей и паразитов>>; и даже его они предпочли бы <<\emph{растрогать}>> и <<переубедить>>.

Когда философы XVIII века гремели против <<привилегий>>, по существу они боролись лишь против \emph{феодальной собственности.} Земельный собственник, в их глазах, был наглый эксплоатор чужого труда, почти бандит. Буржуазная же собственность, напротив, являлась им в совсем благоприятном свете. Коммерческая и промышленная прибыль казалась им продуктом труда коммерсанта и фабриканта; \emph{тайна прибавочной ценности оставалась для них непроницаемой.} Буржуазные теоретики XIX века весьма кстати унаследовали эту теоретическую ошибку своих предшественников. Если доход рабочего далеко не так велик, как капиталиста, это лишь потому, что рабочий не работает или не работал столько, сколько капиталист. Отождествляя прибыль предпринимателя с заработной платой рабочего, Сен\=/Симон и сен\=/симонисты только повторяли ошибку интеллектуальных представителей буржуазии. В теории положение рабочего по отношению к хозяину и следовательно,\---положение пролетариата по отношению к буржуазии, становится ясным и очищенным от всяких заблуждений только с того времени, когда экономическая наука могла наконец объяснить происхождение и природу прибавочной ценности. Это открытие, сделанное Карлом Марксом, положило конец всем ошибкам социалистов в понимании \emph{классовой борьбы.} Социалисты наших дней охотно примут столь дорогой социалистам\=/утопистам проект <<\emph{обратить в свою веру}>> и <<растрогать высшие классы>>\---но при условии: <<обратить>> и <<тронуть>> их \emph{после того, как они будут экспроприированы.} Всякий, кто знает <<человеческую природу>>, согласится, что тогда они гораздо легче <<обратятся>>, чем теперь. \footnote{<<Одни благодаря своему уму, хорошему поведению создают себе капитал и вступают на путь благоденствия и прогресса. Другие, ограниченные, ни ленивые, или развращённые, остаются в стеснительных и трудных условиях существования, основанного единственно на заработной плате>>. Гизо <<О демократии во Франции>>, стр.\=,76.})

Социалисты наших дней хорошо знают, что раз дело идет о борьбе с \emph{арестократией,} какого бы сорта она ни была,\---здесь не может быть речи ни о мире, ни об отдыхе до тех пор, пока она не побеждена и не обезоружена.

Буржуа наших дней обвиняют социалистов в разжигании войны там, где нужно успокаивать и примирять. Они утверждают, что буржуазия никогда подобным образом недействовала. Мы им ответим, как некогда Гизо ответил дворянству; <<\emph{Вырождающаяся раса, история налицо, чтобы вас пристыдить}>>.

<<Контр\=/революция всегда прекрасно понимала, что для достижения своих целей, она своей первой заботой должна была ставить повсеместный захват власти, чтобы вслед за этим её организовать и использовать в своих интересах. Пусть национальная партия в свою очередь знает, что ей важно не разрушить власть, а ее захватить>>.

Так писал Гизо в 1820\=,г. Пока социалисты смешивали экономические интересы пролетариата и буржуазии, они могли иметь лишь ошибочное представление о политическом долге рабочего класса. <<Что касается до так называемых \emph{политических прав}>> писал один из сен\=/симонистов в 1830 году, <<то мы не видим, что общего между ними и благосостоянием масс>> \footnote{Le Globe \No 183.}). Социалисты наших дней, которые не заблуждаются более насчёт непримиримого антагонизма интересов пролетариата и буржуазии, прекрасно видят каким образом <<права, называемые политическими>> связаны с благосостоянием масс. Они понимают, что \emph{всякая классовая борьба\---есть борьба политическая} и они также стараются не уничтожить политическую власть, как это хотели бы <<товарищи\=/анархисты>>, а \emph{захватить её в свои руки.}

Вся история цивилизованного общества состоит из борьбы классов. Французские историки времён Реставрации знали это как нельзя лучше и не забывали до тех пор, пока могильщик буржуазии,\--- современный пролетариат,\---не появился на политической сцене. Но как объяснили себе эти историки тот исторический процесс, который порождает антагонизм интересов в первоначально\-/однородном обществе? Читатель уже видел, что они связывали борьбу третьего сословия против дворянства во Франции с завоеванием галлов франками. Вообще завоевания играют большую роль в их филосовии истории современных народов. Огюстэн Тьерри рассказывает, что однажды, читая некоторые главы из Юма, дабы <<подкрепить>> свои полигические взгляды, он был поражён идеей, которая явилась ему, как луч света и он воскликнул, закрывая книгу: <<\emph{Всё это} происходит от \emph{завоевания,} в основе всего лежит \emph{завоевание}>>. И тотчас же он придумал проект переделать историю революцию в Англии с этой новой точки зрения.\footnote{<<Десять лет исторических изысканий>>, том VI Полного собрания сочинений Огюстена Тьерри. Предисловие.})

Это было в 1817\=,г. С этого временя новая идея нашего автора послужила ему основанием для многих других исторических изысканий; но его <<Очерк революций в Англии>>, изданный в 4\=/м томе <<Европейского критика>> 1817\=,г., ясно показывает как всю ценность так и все слабые стороны его точки зрения.

<<Всякий тот, чьи предки оказались принадлежащими к армии завоевания, покидал свой замок и отправлялся в королевский стан, где принимал командную должность, соответствующую его титулу. Жители городов и портов толпами шли в противоположный лагерь. Можно сказать, что лозунгами этих двух армий были: с одной стороны\--\emph{праздность и власть,} а с другой\---\emph{труд и свобода,} ибо праздные, к какой бы касте они ни принадлежали, те, кто желал в жизни лишь наслаждений без труда,\---вступали в королевскую армию, где они защищали интересы, соответствующие их собственным; в то время как семьи из касты прежних завоевателей, захваченные промышленностью, присоединялись к партии коммун>> \footnote{Очерки революции Etc. Полн. собр. сочин. Огюст. Тьерри, том VI, стр.\=,66.}).

Вот что, таким образом, представляло из себя революционное движение в Англии в 17\=/м веке. Бурная реакция прежних побеждённых против прежних победителей. На первый взгляд это кажется весьма правдоподобным. Но, когда перечитываешь указанный отрывок, является сомнение. Там были потомки прежних победителей, которые, будучи захвачены промышленностью, присоединялись к партии <<труда и свободы>>. С другой стороны королевский лагерь наполнялся всеми теми, кто желал только <<наслаждения без труда>>. И между ними находились, по словам нашего историка, люди всех <<каст>>. Было же здесь стало\-/быть расхождение интересов, в котором большую роль сыграло экономическое движенье, вызванное прогрессом <<промышленности>>. Впрочем Огюстэн Тьерри об этом сам говорит: <<Война велась и той и другой стороной именно за эти положительные интересы. Остальное было лишь внешностью или поводом. Те, кто отстаивал дело подданных, были в большинстве пресвитериане, т.\=,е. они не желали никакого подчинения даже в религии. Те, которые поддерживали противоположную партию, были англиканцы или же паписты, ибо они стремились к власти и взиманию налогов с людей даже в области религиозных культов \footnote{Jbid\---та же страница.}).

Дело, таким образом, совершенно ясно. Борьба велась за \emph{экономические} интересы партий и самая \emph{власть} была по существу лишь орудием, которым эти партии старались завладеть, в целях торжества их интересов. Огюстэн Тьерри понимал это так же хорошо, как и Гизо \footnote{Гизо. \emph{История революции в Англии.} В предисловии автор с большей проницательностью объявляет поверхостным и легковесным мнение, согласно которому революция в Англии была скорее политическое, в то время как французская стремилась преобразовать и правительство и общество. <<Направление ее,\---говорит он,\---было такое же, как и её происхождение>>. Английская революция берёт начало из изменений, происшедших в <<социальном ноложении и нравах английского народа>>. Стр.\=,11\==12 1\=/го тома (издание 1841\=,г.) и речь об истории революции в Англии. Берлин 1850.}) Это не всё. Он понимал также, что вторгаясь в Англию, норманы ставили перед собой определенную экономическую цель: они желали <<\emph{приобресть}>> gaagnier, как говорит он, воспроизводя выражение одного старого рассказчика. Он цитирует речь, произнесенную Вильгельмом Завоевателем перед битвой при Гостинсе, которая показывает нам скрытую подоплеку завоевания \footnote{<<Сражайтесь храбро, воскликнул он, обращаясь в своим друзьям убивайте всех, ибо если мы победим, мы все разбогатеем; что приобрету я, приобретете и вы; если я возьму землю,\---вы будете её иметь>>. (История завоевания Англии норманами. Париж 1838, том 1\=/й, стр.\=,352). С другой стороны, те, на кого нападали, говорили между собой; <<мы должны сражаться какова бы ни была для нас опасность, ибо дело не в том, что мы получим нового господина, а совсем в ином. Герцог норманский отдал ваши земли своим баронам, своим рыцарям, всем этим людям\---и большая часть дала уже ему клятву; они все захотят иметь свою долю; герцог станет нашим королём, и сам будет склонен отдать им наше имущество>>, и т.\=,д. Jbid, стр.\=,347.}). Зачем же ему было аппелировать к завоеванию там, где оно, будучи далеко не в состоянии дать окончательного объяснения явлениям, в свою очередь по своей цели, а \emph{особенно по своим результатам,} обгоняется социальным положением победителей и побежденных?

Дело в том, что школа, к которой принадлежал Огюстэн Тьерри, имела весьма смутные представления об \emph{экономической истории человечества.}

Так же, как и буржуазные экономисты, они считали \emph{капиталистическое общество} единственным, соответствующим \emph{человеческой природе} и воле \emph{Провидения.} Всякая общественная организация, которая не основывалась на капитализме, им казалась противоестественной и по меньшей мере странной (bizarre) \footnote{Странным Огюстен Тьерри называет институт рода и старых британских племён. По Гизо\---<<всегда и везде были и будут существовать рантьери, предприниматели и наёмные рабочие. Эти различия вовсе не случайные или специфические явления, присущие той или иной стране; это явления всеобщие, которые естественно воспроизводятся во всяком человеческом обществе. И чем ближе присматриваешься, тем более убеждаешься в том, что эти явления с одной стороны находятся в тесной связи и в глубокой гармонии с природой человека, которую нам дано познать, а с другой\---с тайными её судьбы, которую дано вам только предвидеть>> (О демократии во Франции, стр.\=,77\==78). Не был ли прав Маркс, говоря, что буржуазные экономисты, как, впрочем, и все теоретики этого класса, знают только два вида учреждений искусственных, либо естественных, и что они в этом сходны с теологами, которые устанавливают также два вида религий: всякая не их религия изобретение людей, в то время как их собственная исходит от Бога? (Нищета философии, стр.\=,113}) Они были способны прекрасно объяснить борьбу средневековой буржуазии с феодальным дворянством, это было движение \emph{естественное,} т.\=,к. оно должно было привести строение общества к типу, продиктованному природой. Но что касается самого феодального строя, то они могли видеть в нём только отклонения исторического движенья от его нормального направления. Наиболее допустимое объяснение подобного отклонения заключалось в насилии завоевателей. Насилие и злоба немного также в <<природе человека>>. Ища в ней основу данной социальной организации, мы не покидаем, таким образом, точки зрения <<человеческой природы>> и одним ударом убиваем двух зайцев; хорошими сторонами человеческой природы мы объясняем капиталистическую систему и всё движение, которое стремится к её установлению; дурными сторонами этой природы объясняем происхождение феодального строя и всякой другой социальной организации, более или менее <<странной>> в глазах буржуа.

Огюстэн Тьерри совершенно так же, как Гизо и Минье, думает, что он поднялся выше исторических взглядов философов предшествующего века, которые видели в средневековья только торжество человеческой глупости, тянувшееся и оборвавшееся. Они претендовали на гораздо большую справедливость по отношению к этой эпохе.

В самом деле, он различил в ней яснее, чем философы 18\=/го века, но то, что он видел, было освободительное движение тогдашних горажан, <<образование и успехи третьего сословия>>, а не <<природа>> фаодального строя, в его целом. Они \emph{понимали феодальный строй в его разложении, но не в его происхождении.} Что касается до происхождения, то <<завоеванье>> не переставало быть для него разрешением загадки.

Мы указали выше, что Ог. Тьерри обязан был Сен\=/Симону всеми своими историческими идеями. Сен\=/Симон придерживался мнения, что и Гизо заимствовал у него свои исторические взгляды. Как бы то ни было, но бесспорно, что тот, кто внимательно прочтет Сен\=/Симона, не найдет в трудах Гизо ничего нового по части философии истории, так и Сен\=/Симон, который настаивал на превосходстве средневековой социальной организации над социальной организацией древних народов, оценивал эти преимущества только с точки зрения того простора, который она давала развитию современного <<промышленного>> строя. Феодализм же для него\---не что иное, как система, основанная исключительно на праве более сильного, состема, в которой господствует дух завоевания \footnote{Единственный важный пункт на котором обычно сходятся современные историки всех наций\---есть нечто иное, как заблуждение. Они всё прозвали века, которые протекли с IX по XII столетия\---веками варварства, а на самом дело это были как раз те века, во время которых установились все те детальные учреждения, которые дали европейскому обществу решающее политическое превосходство над всеми обществами, ему предшествовавшими. <<Memoire sur la qravitation universelle>>, в <<Oeuvres de Saint\=/Simon et Enfentin. Средневековье это эпоха, когда война была и должна была быть рассматриваема как первое средство процветания наций>> и где поместная собственность\dots была по происхождению природы своей чисто военной>>. <<Organisateur>>, собр. соч. т.\=,XX, стр.\=,81 и 83.}).

Бесспорно, что смысл исторического бытия феодальных синьёров заключался прежде всего в их военной функции. В этом смысле можно говорить о военном характере их собственности. Не нужно, однако, забывать, что такое суждение не больше, как facon de parler. Почему в Европе с сегодняшнего дня военная служба проходит иначе чем в средние века? Почему она изменила свою <<природу>>? Потому что экономическая структура европейских обществ не та, какою она была в то время. Способ производства, господствующий в обществе, определяет в последнем счете способ удовлетворения общественных потребностей.

Сколько бы историки той школы, о которой мы здесь говорим, ни повторяли вслед за Минье, что феодализм заключался в потребностях раньше, чем он оказался в действительности, всё же они понимали его <<природу>> так же мало, как и происхождение потребностей общественного человека в зависимости от различных фазисов его эволюции. Их философия истории сводилась к следующему: раньше, чем стать причиной, политические конституции являются следствием; корень (этих конституций) находится в социальном состоянии народов. Социальное состояние определяется состоянием собственности, а у современных народов\---преимущественно состоянием земельной собственности \footnote{Минье. De la Feodalit\'{e}, стр.\=,35 и особенно Гизо <<Essais sur l`histoire de France: изучение состояния земель должен, значит, предшествовать мучению состояния личностей, чтобы понять политические учреждения, нужно быть знакомым с различными социальными условиями и их взаимоотношениями. Чтобы понять эти различные социальные условия, нужно знать природу и отношения собственности (стр.\=,75, 76, 1\=/е изд.)>>. Сравните с Сен\=/Симоном: <<Закон, который образует собственность\---самый важный из законов. Это тот закон, который служит основой социального строя>>.}). Наконец, что касается собственности, она обгоняется природой человека или более или менее сильным искажением этой природы.

Природа человека, которая уже в XVIII ст. играла столь значительную роль в политических и социальных теориях философов и которую О.\=,Конт, мнимый враг метафизики, сделал настоящей сущностью своей будто бы <<социологии>>, не больше, как риторический образ. Неизменна ли человеческая природа? В таком случае не она может объяснить нам изменения, происходящие в общественных отношениях, совокупность каковых изменений образует то, что мы называем историческим процессом. Изменяется ли она в свою очередь? Тогда нужно найти причину этих изменений. В обоих случаях <<природа человека>> одинаково далека от того, чтобы объяснить что бы то ни было в историческом движении человечества.

<<Отношения собственности>> у австралийских племён не похожи на те, которые существуют в настоящее время у народов Западной Европы. Чем это объясняется? Тем ли, что австралийцы имеют, природу отличающуюся от природы европейцев, или тем, что они противятся голосу природы? Ни тем ни другим. Их отношения собственности являются такими, какими они должны быть при нынешнем состоянии \emph{их производительных сил.} Они \emph{естественны,} поскольку они остаются в соответствии с этим состоянием. Они сделаются \emph{противоестественными} тогда, когда производительные силы австралийских племен достигнут более высокого уровня развития.

Для того, чтобы существовать, человек должен воздействовать на внешнюю природу, он должен \emph{производить.} Действие человека на внешнюю природу определяется в каждый данный момент его средствами производства, состоянием его производительных сил: чем больше эти силы, тем продуктивнее их действие. Но развитие производительных сил приводит неизбежно к известным переменам в отношениях производителей друг к другу, в процессе производства. Это те изменения, которые на юридическом языке называются \emph{изменениями состояния собственности.} А так как изменения в состоянии собственности приводят к изменениям во всей общественной структуре, то можно сказать, что развитие производительных сил изменяет <<природу>> общества, и так как, с другой стороны, человек есть продукт окружающей его социальной среды, то очевидно, что развитие производительных сил, изменяя <<природу>> социальной среды, изменяет <<природу>> человека. Природа человека (таким образом)\---не причина, а только следствие.

Если бы, с этой точки зрения, которая есть точка зрения \emph{философии исторического материализма,} захотели разобрать основную историческую концепцию Гизо, Минье и Огюстэна Тьерри, нужно было бы сказать:

Совершенно правильно то, что раньше, чем стать причиной, политические конституции являются следствием; также правильно то, что для того, чтобы понять политические учреждения, нужно знать различные социальные условия и их взаимоотношения, очень правильно и то, что для того, чтобы понять различные социальные условия, нужно знать природу и отношения собственности. Но состояние собственности имеет гораздо большее социальное значение, чем то, которое придавали ему наши историки. Это состояние дает себя чувствовать \emph{везде и не} только у \emph{современных народов;} неправильно также утверждать, что характер политических учреждений определяется главным образом природой земельной собственности; влияние того, что называют движимой собственностью, не менее значительно. Если в средние века крупные земельные собственники составляли господствующий класс в обшестве, это вытекало из состояния производительных сил того времени. Наконец, причину исторического развития форм собственности нужно искать не в природе человека, а в развитии производительных сил.

Мы приходим, таким образом, к выводу, который для многих читателей, предубежденных против материалистического понимания истории, покажется довольно неожиданным. Вывод этот сводится к следующему: исторический материализм Карла Маркса но осуждает поголовно и без разбора исторические идеи предыдущих школ; он только освобождает эти идеи от пагубного противоречия, благодаря которому эти идеи не могли выйти из заколдованного круга.

Другой вывод, который нам кажется не менее достойным внимания\---это то, что если неправильно утверждать, что Маркс был первым заговорившим о классовой борьбе, то всё же вне сомнения, что он первый раскрыл настоящую причину исторического движения человечества и, тем самым, <<природу>> различных классов, которые один за другим появляются на мировой арене. Будем надеяться, что пролетариат сумеет хорошо воспользоваться этим ценным открытием великого мыслителя\-/социалиста.

\begin{flushright}
 \textbf{Перевод А.\=,Ч. и Е.\=,С.}\hspace*{2em}
\end{flushright}
