\section*{\textsc{трибуна}}
\addcontentsline{toc}{section}{\textsc{\textbf{трибуна}}}
\label{sec:9}

\subsection*{О курсах по изучению марксизма при Социалистической Академии}
\addcontentsline{toc}{subsection}{9. Партиец\---О курсах по изучению марксизма при Социалистической Академии}
\label{subsec:9.1}

Одновременно с потребностью поднять уровень марксистского развития массы членов партии, жизнь выдвивула настоятельную необходимость усиления теоретической подготовки и руководящих активных партийцев. С этой целью 10\-/й Съезд Р.К.П., по предложению т.\=,Рязанова, вынес постановление об организации специальных курсов при Социалистической Академии Ц.К.Р.К.П., организуя эти <<систематические курсы по теории, истории и практике марксизма>>, как они названы в постановлении Съезда, ставил условием для 50\-/ти кандидатов: 1) предпочтительность дооктябрьского партийного стажа; 2) не только практический опыт, но и проявление интереса к разработке программных и теоретических вопросов.

На курсы набирались, как видно, товарищи, имеющие не только значительный партийный опыт, но и <<восприимчивые к теории>>, по выражению Энгельса. Здесь в течение двух лет, как научные сотрудники Социалистической Академии, в <<регулярных занятиях по выработанной программе>>, они должны были обрести умение уже самостоятельно в дальнейшем продолжать теоретическую работу. Такие задачи ставились перед этой первой высшей марксистской лабораторией партии.

Пока что этой чрезвычайной важности первый опыт не совсем ещё удался. Причины следующие: 1) слишком пестрый и разнородный состав научных сотрудников; 2) трудность привлечения нужных руководителей и 3) насколько беспорядочный характер и громоздкость программы и занятий.

В партийном отношении научные сотрудники в общем представляют весьма ценный и доброкачественный материал. Но по партийному и общему развитию, по наклонностям и запросам, среди них слишком много различных и даже противоположных элементов. Вот некоторые характерные цифровые данные о научных сотрудниках, которых сейчас на курсах 60 человек:


\begin{table}[h]
\centering
\resizebox{\textwidth}{!}{%
\begin{tabular}{|c|c|c|c||c|c||c|c|c||c|c|}
\hline
\multicolumn{4}{|c||}{1. Партийный стаж} & \multicolumn{2}{c||}{2. Социальн. состав} & \multicolumn{3}{c||}{3. Возраст} & \multicolumn{2}{c|}{4. Образование} \\ \hline\hline
\begin{sideways}До 17 года.\end{sideways}& \begin{sideways}С 17 года.\end{sideways}&\begin{sideways}С 18 года.\end{sideways}&\begin{sideways}С 19 года.\end{sideways}&\begin{sideways}Рабочих.\end{sideways}&\begin{sideways}Интеллиг.\end{sideways}&\begin{sideways}До 20 лет.\end{sideways}& \begin{sideways}21\=,г.\--25\=,л.\end{sideways}&\begin{sideways}Старше 25\=,л.\end{sideways}& \begin{sideways}Низшее и домашнее.\end{sideways}&\begin{sideways}Среднее и высшее.\end{sideways} \\ \hline
14     &19     &20     &7     &15           &45           &3       &27       &30       &10           &50           \\ \hline
\end{tabular}%
}
\end{table}

В общем среди них на курсах можно наметить три основных группы:

1. Первая группа\---это профессиональные партийцы, получившие хорошую партийную выучку на долгой практической работе. Сейчас, в связи с новым этапом революции, они задались целью изучить ряд основных (актуальных) теоретических вопросов, рассчитывая таким образом несколько восполнить свои пробелы в знаниях и подготовиться к дальнейшим самостоятельным теоретическим занятиям уже на практической работе. Это в большинстве своем старые дореволюционные члены партии, половина из них рабочие; таких\--- человек 20.

2. Вторая группа\---это сравнительно молодые партийные товарищи с небольшой партийной выучкой. Они несколько далеки от внутренней жизни партии, мало интересуются актуальными проблемами. Больше питают склонность к академической научной работе, занимаются наукой <<отвлеченно>>, как таковой. Таких\---человек 10.

3. Третья группа\--- это преимущественно боевой молодняк и несколько старших товарищей, которым надо попросту подучиться и получить недостающее им общее и марксистское среднее развитие.

Долго топтались на месте и научные сотрудники, и руководители вместе с инициатором и основателем курсов Д.\=,Б.\=,Рязановым, пока на последних начались занятия. Намеченные на курсах занятия при таком положении вещей полностью начаться, понятно, не могли. Первый год решено было поэтому сделать подготовительным к намеченным курсам. Была избрана линия <<на середняка>>: метод занятий\--- средний между семинарским и лекционным, план\---нечто среднее между предполагавшейся программой и таковой обычного факультета общественных наук. Задача\---дать возможность восполнить знания в области политической экономии, исторического материализма, истории и истории общественной мысли России\-/Запада.

Понятно, что столь обширно намеченные занятия, рассчитанны на небольшой срок времени, не могут носить систематический и выдержанный характер и потому несколько сумбурны. Понятно также, что такого рода занятия при данном составе научных сотрудников приносят им отнюдь не максимальную пользу. Поэтому отрадно при этом отметить, что последние интенсивнейшим образом занимаются самостоятельно по 8\--10 часов в день. Следует только пожелать, чтобы товарищи внесли в самостоятельные свои занятия возможно больше системы и расчета.

Следует также заблаговременно предупредить при определении состава курсов повторение имевших уже место осенью весьма неприятных инцидентов, вносящих, как показал опыт, неизбежное расстройство и сумятицу в ход занятий. Курсы должны быть в будущем году освобождены от захлестнувшей их <<школьной стихии>>: для чисто учебных целей есть ряд гораздо более подходящих учебных заведений, <<курсы же по изучению марксизма>>, во исполнение постановления 10 Съезда Р.К.П., должны иметь более или менее однородный состав вышколенных и развитых партийцев, которым они должны особо приспособленной для этого системой занятий дать помимо интереса и уменье приступить <<к разборке программных и теоретических вопросов>> так, чтобы партия могла из их среды получить отнюдь не профессоров, а марксистски квалифицированных практиков, политиков, теоретиков. Иначе курсы не выполнят возложенных на них партией задач, превратятся в обычную партийную школу высшего типа и потеряют всякий смысл самостоятельного существования. Такой печальный финал курсов нанёс бы большой ущерб насущным интересам партии. И надо поэтому надеяться, что руководители курсов при ряде имеющихся благоприятных условий не дадут этому случиться.

\begin{flushright}
 \textbf{Партиец.}\hspace*{2em}
\end{flushright}

\subsection*{Надо заострять революционное оружие}
\addcontentsline{toc}{subsection}{10. А.\=,Френкель\---Надо заострять революционное оружие}
\label{subsec:9.2}

Большинство членов Р.К.П. мало задумывались над вопросами философии марксизма. Однако, без внимательного изучения философии Маркса и Энгельса нельзя быть действительным коммунистом. Не изучивший сущности материалистической диалектики имеет всегда обыкновение и <<некую страсть>> <<дополнять>> марксизм собственной или позаимствованной философской отсебятиной. Яд реакционного идеализма, обывательщина мещанского эклектизма отравляют и путают неискушенных в области философии марксистов.

Примеров подобного рода пересмотров, путаницы, <<исправлений>> история революционного движения даёт очень много.

Не мало таких попыток было и среди русских марксистов. Пожалуй наш марксизм в значительной мере перещеголял западноевропейский.

Но параллельно с этим в рядах нашего марксизма развивалась и окрепла философия Маркса.

Наша марксистская философия имеет такого большого представителя в России, как Г.\=,В.\=,Плеханов. Он на русской почве создал лучшую литературу по философии марксизма. По ней учились, действовали и действуют основоположники коммунистической партии, по ней должно учиться коммунизму и всё молодое поколение.

Об этом прекрасно пишет В.\=,И.\=,Ленин: <<\dots\emph{нельзя} стать сознательным, \emph{настоящим} коммунистом без того, чтобы не изучать\--- именно изучать всё, написанное Плехановым по философии, ибо это лучшее во всей международной литературе марксизма>> \footnote{В.\=,Ленин. Ещё раз о профсоюзах, текущем моменте и об ошибках т.\=,Троцкого и т.\=,Бухарина, стр.\=,22}).

Следует мимоходом отметить, что, обращая эти слова <<к молодым членам партии>>, тов. Ленин говорит им по поводу <<ошибок>> таких ветеранов партии, как Бухарин и Троцкий. Иногда, как видно, и старые, испытанные члены партии грешат в политике, <<шалят>> на почве <<уклонов>> в области философии. Этот разительный и исключительный пример с тем большей настойчивостью должен подтвердить каждому коммунисту о неотложной необходимости тщательного изучения марксовского материализма.

Подлинным коммунистом, способным к победоносным боям и удачным отступлениям, к революционному подполью и государственному строительству можно стать только овладев и хорошо владея <<основой основ>>, диалектическим методом марксизма, усвоив и впитав его в кровь и плоть свою.

Как бы смело ни разило революционное оружие пролетарского энтузиазма и энергии, победоносным и непобедимым оно становится только тогда, когда заострится материалистическою <<душою>> марксизма, когда материализм, действительно, становится точкой зрения коммуниста и пролетария его философии.

Надо тщательно изучать философию марксизма, точить и заострять своё революционное оружие: впереди ещё много трудной и сложной работы, тяжелой и длительной борьбы.

\begin{flushright}
 \textbf{А.\=,Френкель.}\hspace*{2em}
\end{flushright}
