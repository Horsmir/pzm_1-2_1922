\newpage
\section*{\textsc{от редакции}}
\addcontentsline{toc}{section}{\textsc{от редакции}}
\label{sec:0}

Нам незачем писать декларацию и статьи, чтобы показать читателю своё идейное лицо: мы идём под знаменем, которое в течение слишком полустолетия победоносно развевалось на самых передовых пунктах борьбы классов, которое было знаменем самой прогрессивной силы нашего времени\---пролетариата,\---мы идем под знаменем ортодоксального марксизма.

Не все объединившиеся вокруг нашего журнала коммунисты, нас объединяет общность философского мировоззрения: мы все\---последовательные материалисты.

Нам кажется, необходимо сказать лишь несколько слов о том, как мы понимаем свои задачи.

Мы не стремимся быть исследователями, издали созерцающими и изучающими ход развития идей, борьбу социальных и классовых сил и тенденций в нашем обществе, мы борцы, наш журнал\---журнал борьбы за материалистическое мировоззрение, наш орган\---орган полемики.

Он возник из законного желания молодой пролетарской интеллигенции осмыслить современность, преодолеть эклектизм, проповедуемый некоторой группой так называемых марксистов, углубить и заострить своё знание критикой идейного разброда, царящего среди буржуазной интеллигенции.

Если дореволюционная, старая гвардия нашей партии так искусно <<делает революцию>>, то именно потому, что вся его жизнь в подполье была школой: на воле\---борьбы, в тюрьме и ссылке\---марксовой теории. Великая революция всколыхнула массы, выдвинула на передовые позиции революции широкие слои рабочих и крестьян, влила в ряды авангарда революции\---Р.\=,К.\=,П.\---сотни тысяч молодых членов, которые в огне и буре революции прекрасно научились бороться, но которым еще нужно пройти закаляющую школу марксовой философии, чтобы стать стойкими, уверенными и несокрушимыми коммунистами.

Война еще не окончена, но молодой рабочий авангард стремится использовать тот временный покой, который наступил для пополнения знаний.

Наш журнал пойдёт всемерно навстречу этой назревшей потребности рабочих, он станет одновременно трибуной для широкого слоя рабочих ныне революцией приобщённых к науке.

Борьба с оппортунизмом, с одной стороны, и пессимизмом, мистицизмом и т.\=,д.\---с другой,\---вот дело неотложной важности.

На баррикадах мы показали всему миру, как хорошо мы умеем критиковать оружием. На страницах нашего журнала нам предстоит доказать, что мы хорошо владеем и другим оружием\---критикой.

Всему рабочему авангарду нужно сплотиться вокруг нашего журнала: давно пора нам\---перед лицом всё растущего разложения в лагере врагов\---развернуть знамя воинствующего материализма.


