\newpage
\section*{Письмо тов.\=,Л.\=,Д. Троцкого}
\addcontentsline{toc}{section}{1. Письмо тов.\=,Л.\=,Д. Троцкого}
\label{sec:1}

\emph{Дорогие товарищи!}

Идея издания журнала, который вводил бы передовую пролетарскую молодежь в круг материалистического миропонимания, кажется мне, в высшей степени, ценной и плодотворной.

Старшее поколение рабочих\-/коммунистов, играющее ныне руководящую роль в партии и в стране, пробуждалось к сознательной политической жизни 10\==15\==20 и более лет тому назад. Мысль его начинала свою критическую работу с городового, с табельщика и мастера, поднималась до царизма и капитализма, и затем, чаще всего в тюрьме и ссылке, направлялась на вопросы философии истории и научного познания мира. Таким образом, прежде чем революционный пролетарий доходил до важнейших вопросов материалистического объяснения исторического развития, он успевал уже накопить известную сумму все расширявшихся обобщений, от частного к общему, на основе своего собственного жизненного боевого опыта. Нынешний молодой рабочий пробуждается в обстановке советского государства, которое само есть живая критика старого мира. Те общие выводы, которые старшему поколению рабочих давались с бою и закреплялись в сознании крепкими гвоздями личного опыта, теперь получаются рабочими младшего поколения в готовом виде, непосредственно из рук государства, в котором они живут, нз рук партии, которая этим государством руководит. Это означает, конечно, гигантский шаг вперед в смысле создания условий дальнейшего политического и теоретического воспитания трудящихся. Но в то же время на этом, несравненно более высоком, историческом уровне, достигнутом работой старших поколений, возникают новые задачи и новые трудности для поколений молодых.

Советское государство есть живое отрицание старого мира, его общественного порядка, его личных отношений, его воззрений и верований. Но в то же время само советское государство еще полно противоречий, прорех, несогласованностей, смутного брожения,\---словом, явлений, в которых наследие прошлого переплетается с ростками будущего. В такую, глубоко переломную, критическую, неустойчивую эпоху, как наша, воспитание пролетарского авангарда требует серьезных и надежных теоретических основ. Для того, чтобы величайшие события, могущественные приливы и отливы, быстрые смены задач и методов партии и государства не дезорганизовали сознания молодого рабочего и не надломили его воли еще перед порогом его самостоятельной, ответственной работы, необходимо вооружить его мысль, его волю методом материалистического миропонимания.

Вооружить волю, а не только мысль, говорим мы, потому что в эпоху величайших мировых потрясений более чем когда бы то ни было наша воля способна не сломиться, а закалиться только при том условии, если она опирается на научное понимание условий и причин исторического развития.

С другой стороны, именно в такого рода переломную эпоху, как наша, особенно, если она затянется, т.\=,е. если темп революционных событий на западе окажется более медленным, чем можно надеяться,\---весьма вероятны попытки различных идеалистических и полуидеалистических философских школ и сект, овладеть сознанием рабочей молодежи. Захваченная событиями врасплох\---без предшествующего богатого опыта практической классовой борьбы\---мысль рабочей молодежи может оказаться незащищенной против различных учений идеализма, представляющих, в сущности, перевод религиозных догм на язык мнимой философии. Все эти школы, при всем разнообразии своих идеалистических, кантианских, эмпириокритических и иных наименований, сходятся, в конце концов, на том, что предпосылают сознание, мысль, познание\---материи, а не наоборот.

Задача материалистического воспитания рабочей молодежи состоит в том, чтобы раскрыть пред ней основные законы исторического развития, и из этих основных\---важнейший и первостепеннейший,\--- именно, закон, гласящий, что сознание людей представляет собой не свободный, самостоятельный психологический процесс, а является функцией материального хозяйственного фундамента, т.\=,е. обусловливается им и служит ему.

Зависимость сознания от классовых интересов и отношений, и этих последних\===от хозяйственной организации ярче, открытее, грубее всего проявляется в революционную эпоху. На её незаменимом опыте мы должны помочь рабочей молодежи закрепить в своем сознании основы марксистского метода. Но этого мало. Само человеческое общество уходит и своими историческими корнями, и своим сегодняшним хозяйством в естественно\=/исторический мир. Надо видить в нынешнем человеке звено всего развития, которое начинается с первой органической клеточки, вышедшей, в свою очередь, из лаборатории природы, где действуют физические и химические свойства материи. Кто научился таким ясным оком оглядываться на прошлое всего мира, включая сюда человеческое общество, животное и растительное царства, солнечную систему и бесконечные системы вокруг неё, тот не станет в ветхих <<священных>> книгах, в этих философских сказках первобытного ребячества, искать ключей к познанию тайн мироздания. А кто не признаёт существования небесных мистических сил, способных по произволу вторгаться в личную или общественную жизнь и направлять её в ту или другую сторону, кто не верит в то, что нужда и страдания найдут какую\=/то высшую награду в других мирах, тот тверже и прочнее станет ногами на нашу землю, смелее и увереннее будет в материальных условиях общества искать опоры для своей творческой работы. Материалистическое миропонимание не только открывает широкое окно на всю вселенную, но и укрепляет волю. Оно одно только и делает современного человека человеком. Он еще зависит, правда, от тяжких материальных условий, но уже знает, как их преодолеть, и сознательно участвует в построении нового общества, основанного одновременно на высшей технике и на высшей солидарности.

Дать пролетарской молодежи материалистическое воспитание\---есть величайшая задача. Вашему журналу, который хочет принять участие в этой воспитательной работе, я от души желаю успеха.

\begin{flushright}
 С коммунистическим и материалистическим приветом\\
 \textbf{Л. Троцкий.}\hspace*{3em}
\end{flushright}

27/II 1922\=,г.
