\newpage
\section*{Проблема продукции и революции}
\addcontentsline{toc}{section}{6. Вл.\=,Виленский (Сибиряков)\---Проблема продукции и революции}
\label{sec:6}

Существует такая сказка, в которой один неудачник с путаной головой, перепутав свадьбу с похоронами, вместо приветствия встретившимся веселящимся провозглашает: <<Канун да ладан!>> Неудачник это делает от чистого сердца, но веселящиеся усмотрели в этом насмешку и больно побили неудачного путаника.

В положении такого путаника в апреле 1917 года оказался не безызвестный экономист Пётр Маслов, когда он по поводу законодательного установления 8\=/мичасового рабочего дня,\---первой реальной победы русских рабочих,\---обратился с <<открытым письмом>> к советам рабочих и солдатских депутатов, в котором он предостерегал рабочий класс, заявляя, что без развития производительных сил, без развития промышленности, это завоевание ничего не даст, так как падение индустрии поведёт к безработице и к уничтожению рабочего класса.

Содержание этого письма очень понравилось буржуазии, которая тогда уже начинала чувствовать, что почва колеблется под её ногами, и в аргументах П.\=,Маслова она надеялась найти средство против надвигающейся опасности. Письмо было широко перепечатано буржуазной печатью, которая усиленно расхваливала П.\=,Маслова.

Но для рабочего класса это <<открытое письмо>> прозвучало тогда насмешливым\---<<Канун да ладан>>. И, пожалуй, вышло так, что П.\=,Маслов оказался с помятыми боками. Не знаем, по чьей вине, но он оказался в колчаковии, в Сибири, и после многих пережитых тяжёлых минут сейчас подаёт о себе весть новой написанной им книгой <<Мировая социальная проблема>> \footnote{Издана в г.\=,Чите 1921\=,г.}), в которой он пытается обосновать свою правоту и историей <<открытого письма>> 1917 года.

Однако, ни та горечь, которой пропитана эта последняя книга П.\=,Маслова, ни те аргументы, заимствованные из трёхлетнего <<критического опыта меньшевизма>>, ни, наконец, те выводы, которые он делает, отнюдь не могут служить доказательством правоты отмеченного нами выше выступления Маслова. И с этой стороны новая книжка П.\=,Маслова выглядит немного комически, но в ней есть много такого, что является чрезвычайно интересным и заслуживающим того, чтобы о ней поговорить.

\begin{center}
 \noindent\textasteriskcentered\ \textasteriskcentered\ \textasteriskcentered
\end{center}

П.\=,Маслов считает, что большинство современных экономистов подменяют науку о \emph{народном} хозяйстве наукой о \emph{частном} хозяйстве. Он считает, что до сих пор главное внимание экономистов обращалось на хозяйственную деятельность капиталиста и на результаты этой деятельности с его точки зрения. По его же мнению, перед политической экономией стоит задача <<найти закономерность в распределении и перераспределении производительных сил, направление, в котором оно происходит, и причины, вызывающие его>>.

Эти мысли П.\=,Маслов высказывал ещё десять лет тому назад в своей книге <<Теория развития народного хозяйства>>, и поэтому они новизны не представляют. Но автор пытается сейчас найти в опыте войны и революции подтверждение необходимости выдвижения этой проблемы, которую он отожествляет с мировой социальной проблемой, от успеха разрешения которой зависит будущее человечества.

Новым также для рассматриваемого автора является стремление использовать опыт русской пролетарской революции и результаты её хозяйственной перестройки на новых началах, для того чтобы подкрепить свою основную мысль, что в основе благоприятного разрешения социальной проблемы должна лежать проблема продукции.

В этом отношении работа П.\=,Маслова представляет как бы теоретическое обоснование того течения, которое на протяжении всех четырёх лет кричало о неприемлемости <<потребительского>> социализма большевиков. Однако, стараясь быть объективным, автор, в конечном счёте, всех современных социалистов берёт за одни скобки и делает вывод, что по существу современное понимание социальной проблемы для всех социалистических партий является пониманием как проблемы распределения и что необходимость процесса накопления национального капитала, как не сознавалось точно так же значение различного характера потребления.

Процесс накопления национального капитала и выяснение форм непроизводительного потребления,\--- два основных момента, два фокуса, вокруг которых вертится всё содержание рассматриваемой нами книги. Выдвигая проблему продукции миллионов мелких хозяйств, пользующихся примитивной сохой, автор возражает против иллюзорных утопий электрофикации, полагая, что для России накопление национального капитала связано именно с этими мелкими хозяйствами и должно пойти по пути длительного накопления при сокращении многочисленных форм непроизвотительного потребления.

Капитализм должен смениться другой, более прогрессивной, с точки зрения трудящихся, формой хозяйства, но какой?\---спрашивает П.\=,Маслов и в своём заключении отвечает:\---очевидно, такой формой организации хозяйства, которая обеспечивает увеличение продукции, т.\=,е. увеличит и процесс производительного труда, накопления и продуктивность индивидуального труда.

Мировая война и ее последствия принесли непримиримые противоречия.\---Производительные силы упали, а непроизводительные затраты (напр., вооружение) не прекращаются. Буржуазия для сохранения своей власти не умеет найти других путей, кроме увеличения непроизводительных расходов на вооружение армий; но и рабочий класс не считается с состоянием производительных сил, стремится к увеличению своей доли, независимо в каком положении находятся производительные силы их страны \--- таков заключительный вывод П.\=,Маслова. А отсюда приговор для русского опыта решения социальной проблемы:

<<Опыт русской революции показывает, что обнищание и обострение социальных противоречий, благоприятное для социальных потрясений, неблагоприятно для решения социальной проблемы>>.

Таково примерно существо содержания книжки П.\=,Маслова.

\begin{center}
 \noindent\textasteriskcentered\ \textasteriskcentered\ \textasteriskcentered
\end{center}

Мы не собираемся отрицать факт обнищания современной России, так же как не собираемся оспаривать существа самой проблемы продукции, которая у нас сейчас поставлена на очередь и о которой речь будет идти ниже; но мы, конечно, не можем считать круг развития русской революции замкнутым и рассматривать её как нечто законченное, дающее основание для окончательных выводов.

Если что и можно считать закончившимся, то, может быть, с некоторой натяжкой, это можно сказать относительно периода вооруженной борьбы русского пролетариата, которая была затяжной, кровавой, очень разрушительной и которая наложила свой неизгладимый след на всё, связавное с этим периодом так наз. сейчас <<военного коммунизма>>, где всё было подчинено задачам военной целесообразности и нуждам военной обороны.

Но вряд ли разрушительный процесс войны можно сочетать с творческими производительными задачами, которые выдвигаются экономистом, ставящим перед собою проблему продукции. Война есть тоже продукция, но только разрушения\---это аксиома. И, очевидно, здесь возможно или лицемерно сокрушаться о бесцельности и ненужности подобного рода разрушений, или принимать это разрушение, как факт, и производить его количественный учет, быть может, только порою принимая меры к возможному уменьшению этого разрушения.

Именно в таком положении на протяжении четырех лет находилась Советская Россия, перед рабочим классом и крестьянством которой была диллема: отбиться от вооруженного врага и быть, наконец, самому себе хозяином, или вместе с поражением получить реставрацию прошлого.

Но, может быть, именно эта реставрация\=/то и несла разрешение проблемы продукции? Нет, конечно, П.\=,Маслов так отвечает на этот вопрос\--- <<Решать вопрос о том, сумеет ли это сделать Советская власть, я не берусь, но должен заметить, что власть генералов Колчака, Деникина и Врангеля оказалась неспособной разрешить проблему продукции>>.

А раз так, то тем более рабочему классу России не следовало было отказываться от борьбы за удержание взятой в октябре в свои руки политической власти.

Сам П.\=,Маслов в своем последнем труде признаёт, что критика капиталистического строя, сделанная Карлом Марксом и другими социалистами, настолько сильна и соответствует действительности, что все попытки многочисленных учёных экономистов бороться и опровергнуть эту критику никаких результатов не дали. Между тем русские рабочие как раз следовали по пути, намеченному марксизмом, который этот путь видел в захвате рабочими власти для осуществления социалистического строя через уничтожение капиталистических отношений, что должно было в конечном счёте привести к улучшению положения рабочего класса и к уничтожению противоречий капиталистического общества.

Пролетариат России в союзе с крестьянством взял в свои руки власть и этим как бы осуществил основную предпосылку теории марксизма. Вместе с переходом власти в руки пролетариата перешли и орудия и средства производства из частных рук в общественную собственность. Но этот захват власти и орудий и средств производства отнюдь не был моментальным и не совершился в один день или даже месяц. Нет, это коренная ошибка наших критиков, этот переходный период борьбы шёл свыше четырех лет и закончился только в тот момент, когда мы отбили последнего врага на наших многочисленных фронтах и поставили вопрос о мирном хозяйственном строительстве.

Вся сумма роста непроизводительных расходов, обрастания безобразными бюрократическими извращениями, система пайкового обеспечения непроизводительных элементов и т.\=,п.\---это по существу военные издержки революции. Без подобного рода издержек, конечно, никакая война не обходится. Прямо или косвенно, но оплачивать приходится не только непосредственные средства борьбы, но и такие отношения как союз, нейтралитет и т.\=,п. Поэтому ясно, что длительный период борьбы со всеми её последствиями лишил ваших противников надлежащей перспективы, а потеряв её, они, подобно П.\=,Маслову, стали оперировать такими экономическими фактами нашего недавнего прошлого, которые с точки зрения экономической теории имеют, конечно, только относительную ценность.

Поэтому, вряд ли в серьёз можно говорить для этого периода о возможности процесса накопления. Очевидно, здесь законен был только один процесс\---процесс перераспределения производительных сил в сторону сужения производства и уменьшения общей продукции, что собственно, и должно было в конечном счёте привести к обнищанию страны.

\begin{center}
 \noindent\textasteriskcentered\ \textasteriskcentered\ \textasteriskcentered
\end{center}

Переход от вооруженной борьбы к возможностям мирного строительства позволил для нас поставить вопрос о проблеме продукции и подойти к ней практически со всей решительностью, свойственной большевикам.

Но было бы, конечно, большим заблуждением считать, что большевикам было органически чуждо понимание необходимости проблемы продукции. Формулируя социальную проблему\---<<Социализм\--- это распределение и учёт>>\---Ленин, конечно, не хуже П.\=,М.\=,Маслова, понимал, что предпосылкой для учёта и распределения должно являться производство. Но, как говорят, всякому овощу бывает своё время. Точно также в условиях осаждённой крепости приходится думать не столько насчет проблематических возможностей откуда\-/нибудь получить что\-/нибудь, сколько о том, как правильнее учесть и распределить имеющееся в реальности.

И если формула <<Социализм\---это распределение и учёт>>\---вообще верна для будущего разрешения социальной проблемы, то, не менее она была верна в качестве практического лозунга в условиях русской действительности 1918\==1920\=,гг., когда \emph{Р.С.Ф.С.Р.} являлась крепостью, находящейся в капиталистическом окружении.

То перераспределение производительных сил, которое мы сейчас производим в сторону производственных задач, может с несомненной ясностью свидетельствовать, что, получив возможность начать восстановление своих хозяйственных сил, мы взяли верный курс, поставив перед собою две задачи: первая\---накопление сырьевых ресурсов, могущих служить базой для развертывания наших производительных сил, и, вторая\--- стимулирование производительности труда, создавая необходимые условия для освобождения его от непроизводительных форм.

Переход от продразвёрстки к продналогу, с предоставлением свободного товарооборота, представляет меру, которая должна стимулировать производительные силы сельского хозяйства, т.\=,е. вести к накоплению сырьевых ресурсов. Переход на хозяйственный расчёт в промышленности есть предпосылка стимулирования нашей фабрично\-/заводской промышленности, что, совместно с свободным товарооборотом нашей кустарной промышленности, кладет первые кирпичи в общий фундамент к общему поднятию развития наших производительных сил.

В отношении отказа от непроизводительных форм труда, сейчас идёт пересмотр и ревизия всего наследства от периода вооруженной борьбы. Принцип хозяйственного расчёта дает твёрдые основания для борьбы с теми хозяйственными извращениями, которые могли у нас вырости на почве жестокой военной необходимости вчерашнего дня.

Но было бы ошибкой думать, что этот огромный процесс перевода на новые рельсы хозяйственной жизни Советской России может совершиться в один день. Конечно, этого быть не может в условиях русской действительности, где превалирует пространство и численность населения,\---этот хозяйственный процесс будет длительным, а преобладающие формы мелкого полунатурального хозяйства должны придать всему этому периоду характер периода капиталистического первоначального накопления, свойственного крестьянским странам.

Эта сложность и длительность процесса несомненно осложняется целым рядом моментов той политической действительности, вне пределов которой нельзя, конечно, рассматривать хозяйственный процесс восстановления производительных сил Р.С.Ф.С.Р.

П.\=,Маслов правильно отмечает, что трагизм капиталистического общества и его хозяина\---буржуазии заключается в том, что для сохранения своей классовой власти у неё нет другого пути, как увеличения непроизводительных расходов на армию. К сожалению, в условиях совремевной действительности капиталистического окружения у первой пролетарской республики на пути к сокращению непроизводительных расходов стоит тоже такое препятствие. Пока мир не обеспечен прочно для Р.С.Ф.С.Р., последней приходится, конечно, мириться с непроизводительными расходами на Красную Армию. Это, конечно, минус для проблемы продукции сегодняшнего дня.

Но, если для буржуазно\-/капиталистического государства это есть обычное положение, то для пролетарского это\---временная вынужденная необходимость, которая может отпасть с переходом хотя бы к милиционной системе вооруженных сил.

Другим моментом, чрезвычайно важным для разрешения проблемы продукции, является отказ от непроизводительных форм личного потребления для капиталистического общества выражающегося в роскоши для немногих. Пролетарское государство имеет возможность в общей массе положить этому предел, что мы имели возможность проверить на опыте нашей вынужденной системы сугубой уравнительности. Во всяком случае пролетарское государство имеет возможность стремиться в большей степени, чем капиталистическое, к наиболее производительному использованию своего дохода с точки зрения общественного хозяйства.

Таким образом, правильность намеченной линии и ряд соображений общего принципиального порядка говорят за то, что Советская Россия может и должна расчитывать на возможность разрешения проблемы продукции.

\begin{center}
 \noindent\textasteriskcentered\ \textasteriskcentered\ \textasteriskcentered
\end{center}

В капиталистическом обществе процесс накопления идёт обычно по линии развёртывания производительных сил, сначала в области обмена и средств передвижения товаров, и только затем, во второй своей фазе, переходит в область производства.

Очевидно процесс первоначального накопления, который должно будет пережить наше хозяйство пойдёт тоже по этой линии. Но уже сейчас не трудно предвидеть, что на ряду с сельским хозяйством и развёртыванием торгового капитала, в общем развитии производительных сил страны должна занять соответствующее место крупная промышленность, которая, хотя и в растерзанном виде, но существует у нас.

Это тем более, что крупная промышленность должна явиться основной базой для пролетарской власти. На основе восстановления крупной промышленности пролетариат может расчитывать на укрепление своих рядов новыми резервами, способными усилить его в борьбе за переустройство современного общества.

Крупная промышленность, и очень значительная в своей основной части\---зданиях, машинах и инструментах существует. Хуже дело обстоит с т.\=,н. переменным капиталом в виде сырья и рабочей силы, испытывающей продовольственные затруднения, т.\=,е. по существу лежащие в том же недостатке сырья\---хлеба. Отсюда и вытекает та связь и зависимость между производительными силами сельского хозяйства и крупной промышленности.

У нас процесс хозяйственного восстановления пошёл одновременно по всем указанным выше направлениям. И если стимулированное к усилению производительности свободным оборотом сельское хозяйство должно реализовать недостающее нам сырье, то хозяйственный принцип самоокупаемости, расчёта\--- выгоды и т.\=,п. должен перевести нашу крупную промышленность к руководству принципами наибольшей экономии, т.\=,е. наименьшей затраты для получения наибольших результатов.

Орудия и средства производства, которыми сейчас в крупной промышленности владеет в Советской России рабочий класс, являются одним из элементов, определяющих продукцию страны и развитие её производительных сил. Другим элементом является живой труд пролетариев, от которых требуется не только производительный труд, но его интенсификация.

Ссылаясь на опыт капиталистического общества и на практику военного периода советского строительства, П.\=,Маслов отмечает, что устранение конкуренции или соревнования, а равно стимула личной заинтересованности доказало неизбежность падения общей продуктивности и понижиния средней производительности труда.

В этом отношении известная доля правды есть, но на наш взгляд основная причина падения производительности заключается в общей изношенности, как орудий и средств производства, так равно и живой силы, которая не в меру если так можно выразиться голодала, холодала и, конечно, была физиологически ослаблена в качестве движущей и производящей силы.

Только подкормив изголодавшихся рабочих, только приведя их, так сказать, в нормальное рабочее состояние можно думать об рационализации труда путём его уплотнения, интенсификации и т.\=,п. Учитывая всё это, Советская власть собственно и ставила себе главной задачей в труднейшие годы военных и продовольственных затруднений\---подкормить рабочих, не дать им погибнуть от истощения, отсюда и появился тот классовый паек, который худо или хорошо но выполнил эту задачу.

\begin{center}
 \noindent\textasteriskcentered\ \textasteriskcentered\ \textasteriskcentered
\end{center}

Необходимость, вытекающая из медленного темпа развития революции на Западе, ставит нас в необходимость рассматривать проблему восстановления производительных сил в рамках Р.С.Ф.С.Р. находящейся в капиталистическом окружении. Но это отнюдь не будет теми национальными рамками, которые ставит себе П.\=,Маслов, когда он говорит о <<национальном капитале>>, о <<национальном доходе>> и т.\=,п.

Р.С.Ф.С Р. сейчас представляет огромную федерацию народов входивших ранее в Российскую империю. О национальной связности здесь мало приходится говорить, как и о каком то едином национальном капитале, всё это не более как условность. И если мы сейчас ставим проблему развития производительных сил России, то мы конечно имеем в виду всю федерацию, рамки которой могут быть сужены или раздвинуты в зависимости от обстоятельств.

С этой точки зрения для нас несколько иначе стоит вопрос о противопоставления <<национального капитала>> иностранному капиталу как это делает П.\=,Маслов, когда он говорит, что национальный капитал особенно в отсталой стране, накопляясь, реализуется в ней в виде орудий и средств производства, тогда как, иностранный капитал вывозит прибыль из эксплуатируемой им страны, чем разумеется задерживает её развитие.

Старая колониальная политика европейского капитала давала возможность ставить вопрос таким образом потому, что европейский Капитал в колониях не встречал на своём пути препятствий. Ни черный, ни желтый материки не могли конечно противопоставить силе этого капитала свою силу способную обуздать хищничество европейского капитала. Но разве может эта прошлая практика быть сравнением с той концессионной политикой на которую указывает П.\=,Маслов? Конечно нет. Здесь в основе лежит другое соотношение сил. Р.С.Ф.С.Р. бедна, она нуждается в материализованном капитале в виде орудий производства; но она сильна и это знают все кому надлежит это знать.

Отсюда следует, что взаимоотношения иностранного капитала с объектом эксплуатации должны быть построены на иных основаниях чем это подсказывала бы старая практика. К сожалению здесь сейчас можно только строить предполагая, так как конкретного опыта пока еще нет. Но несомненно, что реальное соотношение сил направит разрешение этого вопроса именно в эту сторону.

Несмотря на замедлившийся темп западно\-/европейской революции для рабочего класса нет конечно оснований терять революционную интернациональную преспективу, которая может сулить расширение завоеваний пролетарской революции, значительно больше рамок нынешней Советской Федерации России, а это, конечно, заставляет под иным углом рассматривать концессионную практику иностранного капитала, которая вне национальных перегородок превращается в простое перемещение и приложение капитала там, где он будет более производителен.

\begin{center}
 \noindent\textasteriskcentered\ \textasteriskcentered\ \textasteriskcentered
\end{center}

Опыт советского строительства народного хозяйства не удовлетворяет П.\=,Маслова. Путает его главным образом обнищание и обострение социальных противоречий, которые мало благоприятствуют разрешению социальной проблемы, отожествляющейся в его понимании с проблемой продукции.

Он видит в капитализме положительные стороны за способность развёртывания процесса производительного труда, накопления и продуктивность индивидуального труда. И если для П.\=,Маслова возможна смена экономических форм то она должна идти по линии более прогрессивной с точки зрения процесса продукции.

Но, когда он обращается к опыту русской пролетарской революции, то этой прогрессивности он не то, чтобы не допускает, а просто сомневается в её возможности. И собственно нам кажется, что его сомнение не столько относится к русскому опыту, сколько к самой природе современной социальной проблеме.

Ему кажется, что в современном понимании \emph{социалистами и рабочими социальной проблемы, как средства значительно и немедленно улучшить положение рабочих при сокращении труда, неизбежно ведёт к сокращению продукции и к падению производительных сил, т.\=,е. как раз к обратным результатам, которые может себе ставить задача разрешения социальной проблемы.}

Получается в своём роде порочный круг из которого экономист П.\=,Маслов не видит выхода.

А отсюда невольно напрашивается мораль, что рабочему классу как будто бы невыгодно заниматься революциями, если он думает так же как П.\=,Маслов серьезно заняться проблемой продукции. Выходит так, что проблеме продукции рабочий класс должен принести в жертву свои классовые интересы и неотъемлемое право на улучшение своего положении, т.\=,е. продать за чечевичную похлебку теоретических измышлений Маслова своё революционное первородство.

Плесенью старого ревизионизма веет от такой новизны и работы П.\=,Маслова. И опять некстати и насмешливо для русского пролетариата звучат сделанные им выводы. Русский пролетариат не отрицает проблемы продукции, он согласен, что ею нужно заняться для того чтобы подвести основание под взятые им политические завоевания. Но он эту задачу решает не путём отказа от революционной стряски, голода, холода, обнищания и тысячи других неприятных последствий всякой борьбы, а именно через борьбу со всеми её последствиями: ибо в этой борьбе он видит основную предпосылку к разрешению социальной проблемны.

\begin{flushright}
 \textbf{Вл.\=,Виленский (Сибиряков).}\hspace*{2em}
\end{flushright}

\section*{О книге г.\=,Сафарова <<Колониальная революция>> \footnote{<<Опыт Туркестана>>\---Гос. Изд. М. 1921\=,г. стр.\=,148.}}
\addcontentsline{toc}{section}{7. В.\=,Ваганян\---О книге г.\=,Сафарова <<Колониальная революция>>}
\label{sec:7}

Книга\--- с большими претензиями.

Автор перед тем, как приступить к изложению тех обобщений, которые делает он на основании опыта Туркестана, посвящает целую главу философии истории востока.

Сафаров вообще весьма расположен к философии (судя но двум его другим брошюрам), но вследствие ли того, что он при всей своей любви к ней\---её не изучал, либо вследствие других причин, но ему она никогда не даётся.

Тут как и в других брошюрах своих у Сафарова масса примитивных и наивных ребячеств, не говоря уж об ошибках. Например, совершенно правильную мысль о том, что капитализм в колониях менее всего ставит себе задачей культуртрегерство, ведёт экстенсивное хозяйство и грабит колонии\---он расписывает в таких выражениях и с такими преувеличениями, что она становится местами прямо ложью.

Так, например, он утверждает, что в колониях <<капитализм не вспахивал почвы для капиталистической фабрики>>\---это прямо не верно: капитализм не ставил себе задачей создавать условия возможности существования капиталистической фабрики, но помимо его воли, логикой вещей там, куда вступает капитализм и начинает хозяйничать\---там водворяются рано или поздно законы капитализма, а тем самым против его желания почва оказывается вспаханной.

Эмпирические факты сегоднешняго состояния Индии\---самой типичной колонии\---тому прекрасное доказательство \footnote{См. статью т.\=,Султан\=/Заде <<К вопросу об индустриализации Индии>>.}), да и Туркестан мог бы послужить хорошей иллюстрацией против Сафарова\---но о Туркестане потом.

<<Тот факт, что на востоке Государственная власть>> как туземная так и пришлая <<выступает непосредственно, как экономический эксплоататор, имеет огромное политическое значение: \emph{никакая политическая революция здесь невозможна без революции экономической}>> (курсив мой\---В.\=,В.).

Это типичный\dots\ бакунизм, как он был воспринят в 70\-/ых годах русскими анархославянофильтрующими народниками хотя они, разумеется не считали
Россию колонией.

Если что и можно вывести из посылки Г.\=,Сафарова, так это то, что этим странам до \emph{экономической} революции недосягаемо далеко, ибо власть выступает непосредственно экономическим эксплоататорам, потому что в стране экономические отношения\---первобытны, производство препримитивное. А раз это так, то <<экономическая революция>> может мерещиться людям воспитанным на традициях и учении Бакунина, а не Маркса.

На самом деле, \emph{социальная революция} в стране, где <<ещё живы кое\-/где остатки родового коммунизма и патриархально родового быта>>! Сафаров говорит <<Первобытная матыга\---\glqq кетмен\grqq и первобытный плут\---\glqq омач\grqq еще до сих пор исчерпывают почти весь технический инвентарь сельскохозяйственного производителя в \glqq Средней Азии\grqq>> (12\==13\=,стр.). <<Целый ряд народностей востока еще не совершили своего окончательного перехода к земледелию (киргизы, арабы, племена Северной Индии и т.\=,д.)>>.

Каково звучит при всём этом афоризм Г.\=,Сафарова?

Если в нём и есть некая доля правды (или лучше было бы сказать возможность правды), то она облечена в такую форму, что стала ложью.

Я бы мог число примеров путаницы в философии истории востока умножить, но думаю и это достаточно: глава не представляет чего\-/то цельного, это плохая непереваренная компиляция и поэтому понять, особенно рабочему, всё это весьма мудрено (скажем и понять\-/то там нечего)!

Ещё менее он гарантирован от ошибок, когда раздражается и нервничает.

А это с ним случается очень часто с III главы, где он. говорит о Туркестане, как колонии.

Русские завоевали Туркестан. Шовинисту туркестанцу приличествует при каждом воспоминании об этом исторгнуть бездну проклятий.

Марксисту же исследователю надлежит хладнокровно и объективно изучить это явление. Сафаров думает иначе. Он думает, что в его задачу входит доказать, что полиция царского правительства\---<<высасывала жизненные соки>>, что купцы <<хищники>>, чиновники\---ташкентцы и т.\=,д. и т.\=,д. Это так часто и навязчиво выступает отовсюду, что чтение становится невозможно нудным.

Однако, преодолеем нудь.

В первой главе он писал, что колонии <<оказались в стороне от технической революции, революционной ломки общественных форм и культурного прогресса>> (стр.\=,3). Однако, когда он от философии перешёл к эмпирии, к разбору туркестанских дел, то оказалось, что <<\emph{приход русских в Туркестан вызвал настоящую революцию в его хозяйственных отношениях}>> (курсив мой\---В.\=,В.) и дальше через страницу: <<колониальный захват Туркестана русскими насильственно подвинул вперёд переход коренного населения от феодализма и феодально\-/патриархального быта к торговому капитализму>>.

Не прав ли был я, когда говорил, что туркестанский опыт будет против самого Сафарова.

Из всех перлов этой главы я отмечу лишь одну странную и упорно повторяемую мысль о <<классовом \emph{неравенстве}>> (курсив его\---В.\=,В.) между нациями.

Во втором пункте тезисов Коминтерна имеется фраза, что коммунистическая партия должна отчетливо разделить нации на <<угнетённых, зависимых и неравноправных>> с одной стороны и <<угнетающих, эксплоататорских, полноправных>>\---с другой.

Однако, эта совершенно правильная мысль как небо от земли далека от развиваемой Сафаровым идеи <<\emph{классового}>> неравенства между нациями.

На самом деле.

Единственный вывод, который можно сделать из <<Тезисов>>, это для победы над капитализмом необходимо <<сближение пролетариата и трудящихся масс всех наций и стран для совместной революционной борьбы за свержение землевладельцев и буржуазии>>, в то время, как посылка Сафарова приводит к совершенно другому выводу. Мы сейчас увидим к какому, а пока разрешите привести выписку, доказывающую, что эти выражения у Сафарова не случайные, а обдуманы, и, как мы увидим несколько ниже, послужили немалой помехой практической деятельности Сафарова и его друзей и единомышленников.

<<Так же как в капиталистическом обществе всё развитие производительных сил совершается в виде усиления господства капитала над трудом, в колониях это развитие только у\=,в\=,е\=,л\=,и\=,ч\=,и\=,в\=,а\=,л\=,о к\=,л\=,а\=,с\=,с\=,о\=,в\=,ы\=,й (разрядка его\---В.\=,В.) антагонизм, классовую противоположность между командующей нацией и нацией угнетенной>>.\---Вот уж воистину сказали дураку богу молись, а он лоб расшиб себе от усердия;\---сказал Коминтерн\---признай существование наций угнетенных и командующих, а Сафаров из излишнего рвения не только это исполнил но и\dots\ <<лоб расшиб>>.

Читатель будет прав, если не удовлетворится этой выпиской: и мысль тут выражена каряво и речь мало связная, да и господь его ведает, быть может в экстазе философствования (а фраза взята именно из главы первой\---стр.\=,4) сказал невпопад что надо.

Вот вторая выписка:

В хлопкоочистительные заводы всю рабочую силу поставляет туземное крестьянство, его беднота, оно и понятно\---эти заводы тесно связаны с господствующей отраслью местного сельского хозяйства\---хлопководством. Это весьма простой и понятный факт, явление далеко не колониальное\---этим сопровождался всякий переход от мельно\-/крестьянской, кустарной промышленности к фабричной (вспомните текстильные фабрики Владимирского района!) и казалось бы если оно что и доказывает, так это то, что не прав Сафаров утверждая, будто капитализм в колониях не <<вспахивал почвы для капиталистической фабрики>>, но логика для него вещь необязательная и он приходит к мысли о том, <<\emph{как классовое неравенство между нациями переносится и в область промышленности}>> (разрядка его\---В.\=,В.).

Если в этой путанице понятий есть какой либо смысл, то это тот, что в колониях рабочие метрополий такие же эксплоататоры туземного пролетариата\---простите ошибся, нужно было сказать, <<класса угнетенных народов>>\--- как и <<их>> буржуазия разумеется вся симпатия и сочувствие Сафарова на стороне туземного <<класса угнетеной нации>>, он иначе не говорит о них, как с величайшим уважением, усугубляя последнее тем, что тут же рабочих метрополий обзывает <<аристократией>>.

<<Тогда как туземный чернорабочий \emph{пролетариат} (разрядка моя\---В.\=,В.) всё время испытывает давление избытка рабочих рук\dots\ европейская рабочая аристократия (не пролитариат!\---В.\=,В.) застрахована от этого самим своим положением>>\---гнев и расстройство нервов\---плохие союзники исследователя!

Но попытаемся без Сафарова и его двух союзников разобраться в вопросе.

Несомненно, ташкентский железнодорожник намного отличается от сарта\---чернорабочего, так же, как бакинский рабочий русский\--- от татарина либо перса. Чем? Прежде всего своим культурным уровнем и большей технической подготовкой.

Материально это выражается в высокой оплате его труда; кроме того у него еще одна отличительная черта\---он организован. Он усиленно старается вовлечь в свою организацию туземцев\---рабочих, но не может, что сильно вредит его работе и ослабляет мощь его организации.

Почему?

По весьма понятным причинам: этот самый полуфеодальный туземный крестьянин, вовлечённый в фабрику ещё не стал пролетарием, он еще целиком в плену у крестьянской идеологии, он еще окончательно непролетаризован.

Этим хорошо пользуются идеологи господствующих классов туземных народов, усугубляя эту рознь, вызванную естественными причинами, искусственной агитацией против <<русских вообще>>, <<англичан вообще>>\--- против <<класса нация \--- угнетатель>> как бы выразился (или должен был бы выразиться) Сафаров.

Посмотрите на наиболее кричащие примеры из Закавказской действительности. Баку самый яркий очаг революции на всём Российском юге, всю тяжесть революционной работы несли на своих плечах рабочие <<аристократы>> и несмотря на это сегодня и тут не мало сафаровоподобных голосов; законная тяга этих рабочих <<аристократов>> к равным себе по развитию российским рабочим квалифицируется как колонизаторство.

Буквально то же самое с рабочими Туркестана: кто вынес на своих плечах революцию 1905\==1907 годов? Железнодорожные рабочие. Правда Сафаров пытается умалить эту борьбу (стр.\=,53), но он тем обнаруживает лишь (при наивыгодных него для предположениях) малое знакомство с историей революции 1905\=,г. ни больше.

Кто вёл неравную борьбу с наседающей со всех сторон контр\-/революцней в течение 1918\==1919\=,гг.? Железнодорожные рабочие.

Как единственный отряд интернационального пролетариата\---железнодорожные рабочие и образовали советскую власть и осуществили диктатуру \emph{пролетариата.} Было бы сугубо пикантно, если туркестанские \emph{националисты} сорганизовали бы \emph{пролетарскую} диктатуру. Конечно, ошибки были неизбежны: клочёк великой пролетарской семьи, окруженный со всех сторон кочевыми народами, осуществляет диктатуру пролетариата: вообразите только эту картину и вы поразитесь геройству, энтузиазму и организованности ташкентских рабочих.

Но это вы, читатели, не Туркестанцы, а вот Сафаров не только не согласен, но даже совсем наоборот:

<<Пролетарская диктатура здесь (т.\=,е. в Туркестане\--- В.\=,В.) с первых же шагов (sic!), приняла типично колонизаторскую (!) внешность: русский рабочий взял на себя \glqq управление народами Туркестана\grqq >> (стр.\=,71).

Сафаров не может понять, что иною и не могла быть диктатура \emph{пролетариата} в стране, где \emph{единственная} организованная пролетарская масса\---русские рабочие. Для того, чтобы этого не получилось и чтобы туземные силы пришли к власти, нужно было <<распролетарить>> диктатуру (да простит мне читатель). Этим делом и занялось второе, <<пришлое>> поколение \emph{россиян} (опять россияне!) в лице уполномоченных центра.

Об этом повествует вся последняя глава и заключение. В этой повести есть интересные моменты и материалы, но не мало и ошибок, извращений и спорных положений. Он до слез возмущён тем, что русские переселенцы занимали земли кочевников под пашню. <<Сгонялись не только удельные киргизские хозяйства, но и мелкие аулы. Сгонялись с насиженных мест, с к\,р\,о\,в\,ь\,ю и п\,о\,т\,о\,м п\,о\,л\,и\,т\,о\,й и з\,а\,в\,е\,щ\,а\,н\,н\,о\,й п\,р\,е\,д\,к\,а\,м\,и з\,е\,м\,л\,и>> (разрядка моя\---В.\=,В.) Пущены в обращение самые \emph{убедительные} аргументы: кровь, пот, <<завещание предков>>,\---особенно последний.

Но добро бы, если эти переселенцы принесли с собой в Туркестан более высокую земледельческую культуру: <<Ничего подобного! Он явился сюда как хищник, и как хищник стал эксплоатировать всё, что у него лежало под рукой>>.

Итак, если вам скажут, что русская земледельческая культура (оседлая) выше киргизской (кочевой и скотоводческой), не верьте,\---это утверждает колонизатор.

Вероятно всё из того же уважения к <<завещаниям предков>>, Сафаров начал проведение коммунизма в Туркестане с того, что выгонял оседлых земледельцев, чтобы открыть дорогу стадам и табунам кочевников. Что и говорить, весьма последовательно (а известное дело, какое похвальное свойство последовательность); однако, в результате получилась картина весьма невесёлая, но об этом в другой раз.

Я не думаю доказывать что\-/нибудь Сафарову (ученого\--- не учат!), я только думаю, что взявшись писать о <<колониальной революции>>, следовало бы избегать манеры плохих чиновников, которые изображают дело так, будто всё до них было катастрофически плохо, а с них начался рай.

Еще один маленький, но весьма характерный дефект: восклицательный и вопросительный знаки часто и в самых различных сочетаниях употребляются для усиления и оттенения мысли, замечания и т.\=,д. Но и тут \emph{мера}\---хорошая вещь. Нельзя же испещрять всю книгу ими и требовать впечатления от них.

До каких курьёзов доходит дело, видно из следующего: в географии имеется термин <<Монгольский Туркестан>>, <<Русский Туркестан>> и т.\=,д. Сафаров хладнокровно не может пройти мимо этого злосчастного <<Русского Туркестана>>, чтобы после слова <<Русский>> не поставить либо восклицательный (!), либо вопросительный (?) знак, либо и того и другого вместе. Теперь всякий уважающий себя <<антиколонизатор>> должен исписать восклицательными и вопросительными знаками весь географический атлас (Японское (!?) море, Земля Франца Иосифа(!!!), и т.\=,д. и т.\=,д.).

При всём том, книжку Сафарова читать следует выбросив, разумеется, первую главу как в интересах читателя, так и автора, а также всю <<философию>> из других глав. Читать нужно не для того, чтобы по Сафарову <<революцию>> в Туркестане возвеличить в <<колониальную революцию>>, а совершенно наоборот,\---чтобы понять, какой революции не следует ожидать в колониях.

Но этот вопрос нас заведёт очень далеко, а нам давно пора кончить.

В ближайших номерах в другой связи мы коснёмся этого вопроса.

\begin{flushright}
 \textbf{В.\=,Ваганян.}\hspace*{2em}
\end{flushright}

\section*{К вопросу об индустриализации Индии}
\addcontentsline{toc}{section}{8. А.\=,Султан\-/Заде\---К вопросу об индустриализации Индии}
\label{sec:8}

Мировая война на время отвлекла внимание великих держав от своих колоний, одновременно ослабив их экономическое давление на них Транспортные затруднения и переход промышленности на военную работу, сильно ограничили вывоз фабрикатов из метрополии и создали благоприятные условия для индустриализации колонии. Правда эти же обстоятельства часто являлись и помехой для её развития, благодаря отсутствию достаточного кадра специалистов и средств производства, но несмотря на все эти трудтости колонии сильно подтянулись в деле насаждения национальной промышленности. В этом отношении особенно большие успехи делала Индия. В этой жемчужине британской короны, почти все отрасли производства, как то: каменноугольная, железная, стальная, химическая, стекольная, текстильная и др. сделали за время воины колоссальный сдвиг вперёд. Из них некоторые раньше совершенно не существовали, как например, химическая и судостроительные верфи для постройки судов, главным образом, для речного и прибережного плавания.

Особенно расцвели отрасли промышленности, связанные с военными потребностями. Так, в 1918\=,г. было сфабриковано 30.000.000 хаки и серых пледов и 49.000.000 ярдов бинтов и др. предметов. До войны эти продукты почти исключительно вывозились из Англии и в самой стране добывалось лишь около 610.000 ярд хаки и 32.000 серых пледов. Наконец, в 1918\=,г. изготовлено 20.000.000 пар сапог, т.\=,е. почти в 20 раз больше чем до войны. Применение туземных дубильных веществ достигло небывалой высоты, учреждены новые кожевни и вызваны к жизни опытные станции для добывания дубильных веществ.

Индийские снабженческие органы распространили свою деятельность, помимо скупки всех предметов, необходимых для армии и железных дорог, также и на поддержку частных торговых фирм, желающих организовать фабрики и заводы, привлекали специалистов и опытных рабочих из Англии и проч.

В 1914\=,г. по официальному отчету, индийские фабрики обработали 4.312.000 кип джута, а в 1918\=,г. 5.447.000 кип, т.\=,е. потребление увеличилось на 25\%. Несмотря на повышение платы рабочим торговые запреты и общие повышения расходов джутовая индустрия Калькутты за это время сильно возрасла. Число предприятий с 64 в 1914 году увеличилось до 76 в 1918 году, а вложенный в них капитал с 42 до 47 мил. долл.

Если индусы часто упрекают англичан в том, что они разрушили промышленность в Индии, превратив ее только в поставщика сырья в индустриальные центры Англии, то этот упрек особеано верен относительно текстильной промышленности. Раньше Индия не только была в состоянии удовлетворить свои ввутренние потребности в мануфактуре, но ещё в больших размерах вывозила. Но в последние годы перед войной её снабжала этими товарами главным образом Англия. Несмотря на противодействие правительства и на $ 3\tfrac{1}{2} $\% налог на местное производство в пользу манчестерских и ланкаширских фабрикантов, за время войны текстильная промышленность в Индии сильно подвинулась вперёд. Число ткацких машин увеличилось с 96.688 в 1914 году с общей стоимостью 274 388 550 фунт. стерлингов до 115.196 в 1918 г. со стоимостью 381.404.170 фунт. стерлингов.

В отношении развития угольной промышленности Индия имеет колоссальную будущность. Её угольные запасы, равняющиеся 79 миллиардов тонн среди азиатских стран, стоят на третьем месте после Китая и Сибири. Благодаря развитию промышленности добыча угля сильно возросла: перед войной она составляла 16 мил. тонн, а в 1918\=,г. она увеличилась почти до 20 мил. тонн. Увеличилась и добыча железа, которая с 372.000 тонн в 1914\=,г. дошла до 492.000 тонн в 1918\=,г. Кроме угля и железа в Индии имеется всё сырье и вспомогательные продукты, в которых нуждаются современные железная и стальная индустрия. Особенно много в ней марганцевой руды, большая часть которой экспортируется в европейские страны.

Благодаря колоссальным залежам каменного угля и железа, успешно развивается и стальная индустрия. Стальные фабрики в Тата к западу от Калькутты заложенные в 1907 году являются самыми крупными предприятиями этого рода в Индии. Во время войны они приобрели ещё большее значение. В 1919 году они вырабатывали в день 175 тонн стали, а позднее 280 тонн. В начале войны там были установлены американские печи, дающие в день 600 тонн стали. Теперь там вырабатываются разные железные и стальные предметы, как например, полосы для пароходов, котлов, мостов, крыш, и товарные вагоны Общество <<Bengal Iron and Steel>> вырабатывавшее раньше 120.000 тонн железа в год теперь вырабатывает свыше 17..000 стали в месяц. До войны имелось лишь две доменные печи, а в 1919 году работали пять крупных заводов, которые вырабатывали 357.000 тонн стали.

Параллельно с этим за время войны возрос и туземный капитал, вложенный в разные предприятия. В 1913\--14\=,г. в Индии было 356 акционерных кампаний, занятых в промышленных предприятиях с капиталом в 675.000.000 рупий. В 1916\--17 году число вновь образовавшихся обществ было 291 с капиталом 250.000.000 рупий. В следующем году образовалось еще 906 компаний, с основным капиталом в 3.000.000.000 рупий. В 1919\--20\=,г. возникло втрое больше компаний с капиталом значительно превышающем предыдущий год.

В общем в Индии сейчас 50\% потребности страны может удовлетвориться фабрикатами внутреннего производства. Мало того, в области текстильной промышленности в некотором смысле становится неприятным конкурентом Англии и Японии. Она до войны вывозила на 20 мил. рупий этих товаров, а в 1916\--17\=,г. вывоз увеличился до 50 мил. Особенно увеличился вывоз крашеных тканей кустарного производства, которое вообще необычайно развилось как вследствие уменьшения импорта, так и вследствие его дороговизны.

Развитие промышленности и усиление туземной буржуазии выдвигают в порядок дня борьбу за национальное освобождение. В палате лордов во время дебатов об Индии (октябрь 1921\=,г.) лорд Чельмоефорд, бывший вице\-/король Индии произнес речь, где он, стараясь не поддаваться панике, тем не менее признал, что превосходство британской расы в Индии теперь оспаривается, против неё соединились все цвета кожи и религии. И действительно экономические интересы перевалируют перед всеми остальными факторами. Мусульманин и индус, до войны враждовавшие между собой, сейчас выступают совместно с требованием экономической и политической свободы. Нет никакого сомнения, что национальная буржуазия Индии при помощи вновь усилившихся кустарей, рабочих и крестьян, одинаково заинтересовано в освобождении Индии от иностранной кабалы, добьется в конце концов полного освобождения от ига английской тирании.

\begin{flushright}
 \textbf{А.\=,Султан\-/Заде.}\hspace*{2em}
\end{flushright}
